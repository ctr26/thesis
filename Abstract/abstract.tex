%!TEX root = ../../thesis.tex
%!TEX enableSynctex = true
% ************************** Thesis Abstract *****************************
% Use `abstract' as an option in the document class to print only the titlepage and the abstract.

\begin{abstract}
    \begin{multicols}{2}\sffamily

Fluorescence microscopy is one of the cornerstones of modern biology but has generally been limited to 2D culture dishes.
Light-sheet microscopy, a recent advance which was awarded Nature Method of the Year 2014, allows fast, non-invasive 3D imaging across an entire organism.
This works by decoupling illumination and detection such that the microscope only illuminates a thin section of tissue at a time. By scanning this \emph{light-sheet} through an organism we can image in 3D, more quickly and with less damage than other techniques such as confocal microscopy.

In this work a custom light-sheet microscope was built which was have applied to studying the cell mechanics of developing embryos. Internal stresses within tissues induce cellular migrations that can govern the organism's resultant anatomy.
A technique was developed to mechanically probe deep tissue using magnetism.
By embedding a magnetic bead in an embryo, we can use a controllable, non-invasive magnetic field to move the bead.
By pushing a magnetic bead and allowing it to relax we can fit a model to its trajectory and so extract local mechanical properties.
The mechanical roles of key proteins in embryonic development can be inferred by comparing results between genetically modified embryos.
Our investigations so far have contradicted previous reports that rho-kinase increases cell stiffness in embryonic development. These results are currently being prepared for publication with our collaborators.


The scope of three-dimensional tracking was moved from going from the microscopic (tracking a single magnetic beads) to nanoscopic scale by to tracking virus particles (virions) which invade host cells and hijack their machinery to replicate and then spread. By visualising the journey of virions through the cell we may reveal weaknesses in infection pathways.
Light-sheet microscopy is better suited than other techniques for tracking virus trafficking in 3D as virions are exceptionally small and fast.
Herpes Simplex Virus 1 was studied which is the widespread cause of cold sores and genital herpes.
Furthermore, it serves as a biological model for other Herpesviruses which are associated with many serious diseases including chickenpox and certain lymphomas.


In addition to designing and constructing a light-sheet fluorescence microscope technological improvement were investigated:
The first was a three-dimensional region-of-interest technique which promises to simultaneously simplify volumetric imaging calibration whilst also being more robust than current approaches.
The second improvement builds upon confocal slit scanning, a technique used to increase image contrast whilst doubling the acquisition time for a single image. This development allowed for full speed imaging with the same increased contrast.



      %  Fluorescence microscopy is one of the cornerstones of modern biology, but has generally been limited to 2D culture dishes.
      %  Light-sheet microscopy, a recent advance which was awarded Nature Method of the Year 2014, allows fast,
      % %  \begin{wrapfigure}[13]{r}[.5\width+.5\columnsep]{6cm}
      % %    \centering
      % %    \begin{subfigure}[t]{0.3\textwidth}
      % %      % \centering
      % %      \includegraphics[width=1\linewidth]{./Abstract/jackson_russell.png}
      % %     {\sffamily \small \emph{Image of beads drifting through the focal plane of the light-sheet microscope where colour encodes time}}
      % %   \end{subfigure}
      % % \end{wrapfigure}
      % % \begin{wrapfigure}[13]{l}[.5\width+.5\columnsep]{6cm}
      % %          \vfill
      % %  \end{wrapfigure}
      %  non-invasive 3D imaging across an entire organism.
      %  This works by decoupling illumination and detection such that the microscope only illuminates a thin section of tissue at a time.
      %  By scanning this `light-sheet' through an organism we can image in 3D, more quickly and with less damage than other techniques such as confocal microscopy.
      %  %we can construct a detailed 3D image with sub-cellular resolution.
      %
      %  %Traditional techniques are slow.
      %  % Super-resolution microscopy allows an improvement in optical imaging resolution previously thought to be physically impossible.
      %  % Super res great, but only good in 2D. %Confocal is 3D but is slow and "force entire" and isn't super res.
      %  % lightsheet is fast, 3D and can be supe res, by seperating objective
      %  %We can now visually inspect live biological specimens in real time and at protein length scales.
      %  %These imaging techniques do however require long acquisition and exposure times, and so force the entire biological sample to be flooded with light, which subjects delicate samples to harmful radiation.
      %  %Light sheet Fluorescence Microscopy is distinct from these techniques in that it introduces an additional excitation lens at right angles to the detection, confining the illumination to the plane of interest and minimising harm to the sample.
      %  %This decoupling then allows for fast, non-invasive 3D imaging across an entire organism.
      %  %This decoupling then allows for fast, deep and non-invasive volumetric imaging.
      %  %Light sheet Fluorescence Microscopy was Nature Method of the Year 2014 and promises to be the future standard, disrupting 400 years of established microscopy.
      %  %These advances have been so revolutionary that Light-sheet fluorescence microscopy was awarded Nature Method of the Year 2014.
      %
      %  % During the first year of my PhD I designed and built
      %  In this work a custom light-sheet microscope was built which was have applied to studying the cell mechanics of developing embryos.
      %  %Cell mechanics plays a vital role in the development of organisms;
      %  Internal stresses within tissues induce cellular migrations that can govern the organism's resultant anatomy.
      %  A technique was developed to mechanically probe deep tissue using magnetism.
      %  By embedding a magnetic bead in an embryo we can use a controllable, non-invasive magnetic field to move the bead.
      %  By pushing a magnetic bead and allowing it to relax we can fit a model to its trajectory and so extract local mechanical properties.
      %  % Comparing results between embryos that are genetically modified to no longer produce different key proteins provides an understanding of their roles in embryonic development. %TODO Reword
      %  The mechanical roles of key proteins in embryonic development can be inferred by comparing results between genetically modified embryos.
      %  %Using light-sheet imaging we can rapidly visualise entire live cells, permitting an unprecedented opportunity to observe and induce cell migration and tissue formation.
      %  %Our investigations so far have demonstrated that rho-kinase, in embryonic development increases cell stiffness
      %  %is inaccurate. %and %TODO add something.+
      %  Our investigations so far have contradicted previous reports that rho-kinase increases cell stiffness in embryonic development.
      %  These results are currently being prepared for publication with our collaborators.
      %
      %
      %  % Looking ahead, I intend to  single magnetic bead tracking to that of single virus particle tracking, going from the microscopic to the nanoscopic.
      %  % Looking ahead, I intend to
      %  The scope of three dimensional tracking was moved from going from the microscopic (tracking a single magnetic beads) to nanoscopic scale by to tracking virus particles (virions) which invade host cells and hijack their machinery to replicate and then spread. %their disease
      %  By visualising the journey of virions through the cell we may reveal weaknesses in infection pathways.
      %  %Light-sheet microscopy is the only technique capable of volumetrically tracking virions, which are both exceptionally small and fast.
      %  Light-sheet microscopy is better suited than other techniques for tracking virus trafficking in 3D as virions are exceptionally small and fast.
      %  Herpes Simplex Virus~1 was studied which is the widespread cause of cold sores and genital herpes.
      %  Furthermore, it serves as a biological model for other Herpesviruses which are associated with many serious diseases including chickenpox and certain lymphomas.
      %  %and life-threatening conditions in immuno-compromised patients.
      %  %Herpesvirus pathogens are ubiquitous in vertebrates and establish life-long latent infections in their hosts.
      %  %Infections by the nine known human herpesviruses are associated with many serious diseases including certain lymphomas and life-threatening conditions in immuno-compromised patients.
      %
      %  In addition to designing and constructing a light-sheet fluorescence microscope technological improvements were investigated:
      %  %In addition to design, construction and application, I am also contributing directly to the field of light-sheet microscopy itself.
      %  %So far I have proposed two improvements.
      %  The first was a three dimensional region-of-interest technique which promises to simultaneously simplify volumetric imaging calibration whilst also being more robust than current approaches.
      %  The second improvement builds upon confocal slit scanning, a technique used to increase image contrast whilst doubling the acquisition time for a single image.
      %  This development allowed for full speed imaging with the same increased contrast.
      %  % I am currently preparing two manuscripts detailing these improvements for submission to \emph{Optics Letters}.
      %  % Herpesviruses also cause a significant disease burden in animals that can lead to economic problems for livestock farmers.
      %  % In visualising HSV1 virions leaving a cell (a step which contributes directly to viral pathology but is lesser studied)
      %  % Specifically, we intend to track HSV1 as it exits a cell, after replication; the spreading stage
      %  % This aspect of viral infection is poorly understood but contributes directly to virus pathology
      %  % Medicines which could block a virion from exiting from a cell would inhibit the viral spread; this medical imprisonment of virions could then provide an effective, curative treatment.
      %  % Virions invade host cells and hijack their machinery to replicate and then spread their disease. By visualising the journey of virions through the cell we may reveal finding weaknesses in its infection pathology.
      %  % Specifically we intend to study Herpes Simplex Virus 1, as it begins to exit a cell and spread.
      %  % By understanding how virions interact with cellular machinery we can provide targeted medicine
      %  % By visualising the journey of virions we may reveal finding weaknesses in its infection pathology
      %  % Light sheet microscopy is the only technique capable of tracking virons which are both exceptionally small and fast
      %
      %  % Light-sheet microscopy is at the cutting edge of live-organism imaging.
      %  %In the coming years it will be adopted as the standard of biological imaging and transform modern biology into a highly quantitative, multidisciplinary and exciting field.
      %  % In the coming years it will help move biology out of the petri dish and back into the animal.
      %  % Further development requires the marriage of physics, maths, statistics, computer science, chemistry and biology, in laboratories such as the Laser Analytics Group.
\end{multicols}
\clearpage
\begin{figure}
  \centering
  \begin{subfigure}[t]{0.7\textwidth}
    % \centering
    \includegraphics[width=\linewidth]{./Abstract/jackson_russell.png}
    \linebreak
    { \sffamily \emph{Image of beads drifting through the focal plane of the light-sheet microscope built during this work, where colour encodes time; featured in Interalia magazine 2016~\cite{MicroChoreographyGallery2016}}}
    % \caption{Image of beads drifting through the focal plane of the light-sheet microscope where colour encodes time}
  \end{subfigure}
\end{figure}
\end{abstract}
