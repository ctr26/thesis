%!TEX root = ../thesis.tex
% !TEX TS-program = xelatex
% ****************************** Misc ******************************************
\usepackage{blindtext}
% ******************************************************************************
% ****************************** Custom Margin *********************************
% Add `custommargin' in the document class options to use this section
% Set {innerside margin / outerside margin / topmargin / bottom margin}  and
% other page dimensions
\ifsetCustomMargin
  \RequirePackage[left=37mm,right=30mm,top=35mm,bottom=30mm]{geometry}
  \setFancyHdr % To apply fancy header after geometry package is loaded
\fi

% Add spaces between paragraphs
%\setlength{\parskip}{0.5em}
% Ragged bottom avoids extra whitespaces between paragraphs
\raggedbottom
% To remove the excess top spacing for enumeration, list and description
%\usepackage{enumitem}
%\setlist[enumerate,itemize,description]{topsep=0em}

% *****************************************************************************
% ******************* Fonts (like different typewriter fonts etc.)*************
\usepackage{ wasysym }

% Add `customfont' in the document class option to use this section

\ifsetCustomFont
  % Set your custom font here and use `customfont' in options. Leave empty to
  % load computer modern font (default LaTeX font).
  \RequirePackage{mathpazo}
  \usepackage{amsmath}
  \usepackage{newpxtext}
  %\usepackage{mathpazo}
  %\usepackage[T1]{fontenc}
  %\usepackage{fontspec}

  \usepackage{sectsty}
  \usepackage{roboto}
  \allsectionsfont{\sffamily} % <---- omitting \bfseries still gives bold font
  %\usepackage{xfrac,fontspec,unicode-math}
  %\setmathfont[version=lm]{Latin Modern Math}
  % For use with XeLaTeX
  %  \setmainfont[
  %    Path              = ./libertine/opentype/,
  %    Extension         = .otf,
  %    UprightFont = LinLibertine_R,
  %    BoldFont = LinLibertine_RZ, % Linux Libertine O Regular Semibold
  %    ItalicFont = LinLibertine_RI,
  %    BoldItalicFont = LinLibertine_RZI, % Linux Libertine O Regular Semibold Italic
  %  ]
  %  {libertine}
  %  % load font from system font
  %  \newfontfamily\libertinesystemfont{Linux Libertine O}
\fi

% *****************************************************************************
% **************************** Custom Packages ********************************

% ************************* Algorithms and Pseudocode **************************

\usepackage{algpseudocode}


% ********************Captions and Hyperreferencing / URL **********************

% Captions: This makes captions of figures use a boldfaced small font.
%\RequirePackage[small,bf]{caption}

\RequirePackage[labelsep=space,tableposition=top,small,bf]{caption}
\renewcommand{\figurename}{Fig.} %to support older versions of captions.sty


% *************************** Graphics and figures *****************************

%\usepackage{rotating}
%\usepackage{wrapfig}

% Uncomment the following two lines to force Latex to place the figure.
% Use [H] when including graphics. Note 'H' instead of 'h'
%\usepackage{float}
%\restylefloat{figure}

% Subcaption package is also available in the sty folder you can use that by
% uncommenting the following line
% This is for people stuck with older versions of texlive
%\usepackage{sty/caption/subcaption}
\usepackage{subcaption}

% ********************************** Tables ************************************
\usepackage{booktabs} % For professional looking tables
\usepackage{multirow}

\usepackage{multicol}
\usepackage{longtable}
\usepackage{tabularx}


% *********************************** SI Units *********************************
\usepackage[allow-number-unit-breaks=true,separate-uncertainty=true,multi-part-units=single,binary-units=true]{siunitx} % use this package module for SI units

% ******************************* Line Spacing *********************************

% Choose linespacing as appropriate. Default is one-half line spacing as per the
% University guidelines

% \doublespacing
 \onehalfspacing
% \singlespacing

\usepackage{enumitem}

% ************************ Formatting / Footnote *******************************

% Don't break enumeration (etc.) across pages in an ugly manner (default 10000)
%\clubpenalty=500
%\widowpenalty=500

%\usepackage[perpage]{footmisc} %Range of footnote options


% *****************************************************************************
% *************************** Bibliography  and References ********************

%\usepackage{cleveref} %Referencing without need to explicitly state fig /table

% Add `custombib' in the document class option to use this section
% \ifuseCustomBib
   % \RequirePackage[numbers,sort&compress]{natbib} % CustomBib

% If you would like to use biblatex for your reference management, as opposed to the default `natbibpackage` pass the option `custombib` in the document class. Comment out the previous line to make sure you don't load the natbib package. Uncomment the following lines and specify the location of references.bib file


\RequirePackage[style=nature,date=year,backend=bibtex,doi=false,isbn=false,url=false,sorting=none,sortcites=true,doi=false,url=false,hyperref]{biblatex}
\addbibresource{./References/references.bib,./References/references.bib} %Location of references.bib only for biblatex, Do not omit the .bib extension from the filename.
\bibliography{References/references,References/_references} %Location of references.bib only for biblatex

% \fi

% changes the default name `Bibliography` -> `References'
\renewcommand{\bibname}{References}


% ******************************************************************************
% ************************* User Defined Commands ******************************
% ******************************************************************************

% *********** To change the name of Table of Contents / LOF and LOT ************

%\renewcommand{\contentsname}{My Table of Contents}
%\renewcommand{\listfigurename}{My List of Figures}
%\renewcommand{\listtablename}{My List of Tables}


% ********************** TOC depth and numbering depth *************************

\setcounter{secnumdepth}{2}
\setcounter{tocdepth}{2}


% ******************************* Nomenclature *********************************

% To change the name of the Nomenclature section, uncomment the following line

%\renewcommand{\nomname}{Symbols}


% ********************************* Appendix ***********************************

% The default value of both \appendixtocname and \appendixpagename is `Appendices'. These names can all be changed via:

%\renewcommand{\appendixtocname}{List of appendices}
%\renewcommand{\appendixname}{Appndx}

% *********************** Configure Draft Mode **********************************

% Uncomment to disable figures in `draft'
%\setkeys{Gin}{draft=true}  % set draft to false to enable figures in `draft'

% These options are active only during the draft mode
% Default text is "Draft"
%\SetDraftText{DRAFT}

% Default Watermark location is top. Location (top/bottom)
%\SetDraftWMPosition{bottom}

% Draft Version - default is v1.0
%\SetDraftVersion{v1.1}

% Draft Text grayscale value (should be between 0-black and 1-white)
% Default value is 0.75
%\SetDraftGrayScale{0.8}


% ******************************** Todo Notes **********************************
%% Uncomment the following lines to have todonotes.

\ifsetDraft
	\usepackage[colorinlistoftodos]{todonotes}
	\newcommand{\mynote}[1]{\todo[author=ctr26,size=\small,inline,color=green!40]{#1}}
\else
	\newcommand{\mynote}[1]{}
	\newcommand{\listoftodos}{}
\fi

% Example todo: \mynote{Hey! I have a note}

%%%
\usepackage{pgfplots}
\usepackage{tikzscale}
\usepackage{helvet}

\usepackage[eulergreek]{sansmath}
\pgfplotsset{
  tick label style = {font=\sansmath\sffamily\footnotesize},
  %every axis label = {font=\sansmath\sffamily},
  %legend style = {font=\sansmath\sffamily},
  %label style = {font=\sansmath\sffamily}
}
\usepackage{roboto}
% https://github.com/matlab2tikz/matlab2tikz/issues/672
\newlength{\figwidth}
\newlength{\figheight}

\setlength{\figwidth}{0.4\textwidth}
\setlength{\figheight}{0.6180\figwidth}


\usepackage{graphicx}
\usepackage{xcolor}
\definecolor{darkblue}{rgb}{0,0,0.5}
\usepackage{transparent}

\usepackage{import}

\usepackage{pdflscape}

\usepackage{dpfloat}
\usepackage{multicol}

\usepackage{mhchem}
\usepackage{wrapfig}
\usepackage[super]{nth}


\newcommand{\nosection}[1]{%
  \refstepcounter{section}%
  \addcontentsline{toc}{section}{\protect\numberline{\thesection}#1}%
  \markright{#1}}

  % \newcommand{\nochapter}[1]{%
  %   \refstepcounter{chapter}%
  %   \addcontentsline{toc}{chapter}{\protect\numberline{\thechapter}#1}%
  %   \markright{#1}}

  \newcommand{\nochapter}[1]{%
    \refstepcounter{chapter}%
    \addcontentsline{toc}{chapter}{#1}%
    \markright{#1}}

  \usepackage{tikz}
  \usepackage{graphicx}
  \usetikzlibrary{positioning}

  \AtEveryBibitem{%
    \clearfield{note}%
  }

%
% \usepackage[active,graphics]{preview}
%
% \usepackage{array}
% \makeatletter
% \font\dummyft@=dummy \relax
% \dummyft@
% \tracinglostchars=\z@
% \count@\sixt@@n
% \loop
% \ifnum\count@ >\z@
% \advance\count@\m@ne
% \textfont\count@\dummyft@
% \scriptfont\count@\dummyft@
% \scriptscriptfont\count@\dummyft@
% \repeat
% \let\selectfont\relax
% \let\mathversion\@gobble
% \let\getanddefine@fonts\@gobbletwo
% \pagestyle{empty}
% \let\ps@fancy\ps@empty
% \let\hline\relax
% \newcolumntype{|}{}
% \let\cleardoublepage\relax
% \makeatother
