%!TEX root = ../thesis.tex
%!TEX enableSynctex = true
%*******************************************************************************
%****************************** Third Chapter **********************************
%*******************************************************************************
% **************************** Define Graphics Path **************************
\ifpdf
    \graphicspath{{Chapter3/Figs/Raster/}{Chapter3/Figs/PDF/}{Chapter3/Figs/}}
\else
    \graphicspath{{Chapter3/Figs/Vector/}{Chapter3/Figs/}}
\fi

\chapter[Light-sheet microscopy combined with remote force measurements]{Light-sheet microscopy combined with remote force measurements}%:\\ \Large Correlating real-time viscoelastic changes with embryonic development}

%From Julien:
How multicellular organisms enact the morphogenetic programmes that ensure their characteristic forms remains an enigma. Genetic screens have yielded an array of essential structural, patterning and signalling pathways with which morphogenesis is orchestrated.
However, morphogenesis is ultimately a physical phenomenon that requires a physical explanation.  In vivo imaging of morphogenesis allows measurements that reveal stereotypical patterns in the cellular behaviour by individual and groups of cells.
These are indicative of active force generation but are insufficient to construct a quantitative explanation of where forces are generated and how forces propagate within and between tissues. To overcome these limitations, we require a quantitative characterisation of the physical properties of the tissues involved.
Only with this knowledge are we able to understand how forces propagate within tissues to bring about morphogenesis. Measurements of the properties of individual cells () and for bulk tissues () have revealed both viscoelastic or visco-elasto-plastic solids ().
[Cell autonomous force generation uses actomyosin-based contractile activity. – need to mention intrinsic forces someplace]   Bulk tissue properties can be estimated using atomic force microscopy to investigate a tissue surface. Alternatively, micropipette aspiration can probe dissociated cells or explants, however deep tissue cannot be assessed directly. More recently, techniques have been developed to measure tissue stress and viscoelastic properties, utilising laser ablation,  oil droplets, or embedding tissue explants in matrix gel.
Most recently, ferrofluid droplets have shown that local tissue properties, that change regarding the tissue localisation. We are still in need of methods that can give a repeated real-time readout of physical properties and relate those measurements to the underlying morphogenetic behaviour. [**What do we say that makes us different to oil droplets?]

We sought a method that can give a non-destructive, quantitative measurement of local tissue physical properties at the length scale of a few cells, completed with seconds to minutes and repeatable over developmentally-significant periods.
We chose to use biologically-compatible paramagnetic beads, implanted into the developing zebrafish embryo
A four-pole electromagnetic device was built that produces a controlled magnetic field gradient in 3D, such that a bead can be moved with known force.
Tracking bead movement gives the dynamic material properties of the surrounding tissue. In a first test of this approach, we have followed the emergence of the first cohesive tissue of the zebrafish blastula, between the “high” to “sphere” stages of development(). After the mid-blastula transition, mesenchymal blastomeres become first motile and adherent to form the tissue that will go on to contribute to the first morphogenetic movement of the embryo.
We could show for the first time that a three-fold elevation of tissue elasticity and viscosity  are associated with this development.
This elevation is dependent upon E-cadherin-based adhesions and Rac-1 dependent cell protrusive activity, abrogating either interfered with these developmental changes. Interestingly, reducing Rho-kinase dependent cell contractility increased both tissue viscosity and elasticity  and raised the number of cell protrusions.
To better comprehend the cellular basis for the physical properties detected by our new method, we combined magnetic tweezers with light sheet microscopy.
Together, this permitted us to correlate a viscoelasticity with cell shape change and a viscosity with cell rearrangements, both cellular changes reduce as tissue elasticity and viscosity increase.
We can now assign the viscoelastic component predominantly to cell shape change and viscosity to cell rearrangement.
[Some of this really belongs in the discussion but will leave here for now]
[If stiffness is explained by cell volume, then effect should be proportional to change in r, r decreases just a few %, while stiffness changes 3x][stress stiffening]
*** how much “plastic” change is rearrangement of extracellular space? ***


\section{Tissue dynamics in developing organisms}
\subsection{Embryonic rheology}
\section{Methods of measuring tissue dynamics}
\section{Magnetic tweezers combined with Light-sheet microscopy}
\subsection{Design}
\section{Results}
\section{Discussion}
