%!TEX root = ../../thesis.tex
%!TEX enableSynctex = true
%*******************************************************************************
%****************************** Third Chapter **********************************
%*******************************************************************************
% **************************** Define Graphics Path **************************
\ifpdf
    \graphicspath{{Chapters/aims/Figs/Raster/}{Chapters/aims/Figs/PDF/}{Chapters/aims/Figs/}}
\else
    \graphicspath{{Chapters/aims/Figs/Vector/}{Chapters/aims/Figs/}}
\fi

\chapter{Introduction}

%
% % \noindent
% %Fluorescence microscopy has experienced a scientific renaissance over the last 20 years through advances in new fluorescent proteins and super-resolution optical microscopy, both of which were awarded with Nobel Prizes in Chemistry.
% Fluorescence microscopy is one of the cornerstones of modern biology, but has generally been limited to 2D culture dishes.
% Light-sheet microscopy, a recent advance which was awarded Nature Method of the Year 2014, allows fast, non-invasive 3D imaging across an entire organism.
% This works by decoupling illumination and detection such that the microscope only illuminates a thin section of tissue at a time.
% By scanning this `light-sheet' through an organism we can image in 3D more quickly and with less damage than other techniques such as confocal microscopy.
% %we can construct a detailed 3D image with sub-cellular resolution.
%
% %Traditional techniques are slow.
% % Super-resolution microscopy allows an improvement in optical imaging resolution previously thought to be physically impossible.
% % Super res great, but only good in 2D. %Confocal is 3D but is slow and "force entire" and isn't super res.
% % lightsheet is fast, 3D and can be supe res, by seperating objective
% %We can now visually inspect live biological specimens in real time and at protein length scales.
% %These imaging techniques do however require long acquisition and exposure times, and so force the entire biological sample to be flooded with light, which subjects delicate samples to harmful radiation.
% %Light sheet Fluorescence Microscopy is distinct from these techniques in that it introduces an additional excitation lens at right angles to the detection, confining the illumination to the plane of interest and minimising harm to the sample.
% %This decoupling then allows for fast, non-invasive 3D imaging across an entire organism.
% %This decoupling then allows for fast, deep and non-invasive volumetric imaging.
% %Light sheet Fluorescence Microscopy was Nature Method of the Year 2014 and promises to be the future standard, disrupting 400 years of established microscopy.
% %These advances have been so revolutionary that Light-sheet fluorescence microscopy was awarded Nature Method of the Year 2014.
%
% %During the first year of my PhD
% A custom light-sheet microscope was designed and built  for the application of cell mechanics of developing embryos and the tracking of viral egress.
% %Cell mechanics plays a vital role in the development of organisms;
% Internal stresses within tissues induce cellular migrations that can govern the organism's resultant anatomy.
% We have developed a technique to mechanically probe deep tissue using magnetism.
% By embedding a magnetic bead in an embryo we can use a controllable, non-invasive magnetic field to move the bead.
% By pushing a magnetic bead and allowing it to relax we can fit a model to its trajectory and so extract local mechanical properties.
% % Comparing results between embryos that are genetically modified to no longer produce different key proteins provides an understanding of their roles in embryonic development. %TODO Reword
% The mechanical roles of key proteins in embryonic development can be inferred by comparing results between genetically modified embryos.
% %Using light-sheet imaging we can rapidly visualise entire live cells, permitting an unprecedented opportunity to observe and induce cell migration and tissue formation.
% %Our investigations so far have demonstrated that rho-kinase, in embryonic development increases cell stiffness
% %is inaccurate. %and %TODO add something.+
% Our investigations so far have contradicted previous reports that rho-kinase increases cell stiffness in embryonic development. These results are currently being prepared for publication with our collaborators.
%
% % Looking ahead, I intend to  single magnetic bead tracking to that of single virus particle tracking, going from the microscopic to the nanoscopic.
% %Looking ahead, I intend to move from tracking single magnetic beads to tracking single virus particles, going from the microscopic to nanoscopic scale.
% Virus particles (virions) invade host cells and hijack their machinery to replicate and then spread. %their disease
% By visualising the journey of virions through the cell we may reveal weaknesses in infection pathways.
% %Light-sheet microscopy is the only technique capable of volumetrically tracking virions, which are both exceptionally small and fast.
% Light-sheet microscopy is better suited than other techniques for tracking virus trafficking in 3D as virions are exceptionally small and fast.
% We intend to study Herpes Simplex Virus~1 which causes cold sores and genital herpes.
% Furthermore, it serves as a biological model for other Herpesviruses which are associated with many serious diseases including chickenpox and certain lymphomas.
% %and life-threatening conditions in immuno-compromised patients.
% %Herpesvirus pathogens are ubiquitous in vertebrates and establish life-long latent infections in their hosts.
% %Infections by the nine known human herpesviruses are associated with many serious diseases including certain lymphomas and life-threatening conditions in immuno-compromised patients.
%
% In addition to designing and constructing a light-sheet fluorescence microscope I have made technological improvements which will be useful for other researchers.
% %In addition to design, construction and application, I am also contributing directly to the field of light-sheet microscopy itself.
% %So far I have proposed two improvements.
% The first is a three dimensional region-of-interest technique which promises to simultaneously simplify volumetric imaging calibration whilst also being more robust than current approaches.
% The second improvement builds upon confocal slit scanning, a technique used in our lab to increase image contrast but takes twice as long to acquire an image.
% This development now allows for full speed imaging with the same increased contrast.
% I am currently preparing two manuscripts detailing these improvements for submission to \emph{Optics Letters}.
% % Herpesviruses also cause a significant disease burden in animals that can lead to economic problems for livestock farmers.
% % In visualising HSV1 virions leaving a cell (a step which contributes directly to viral pathology but is lesser studied)
% % Specifically, we intend to track HSV1 as it exits a cell, after replication; the spreading stage
% % This aspect of viral infection is poorly understood but contributes directly to virus pathology
% % Medicines which could block a virion from exiting from a cell would inhibit the viral spread; this medical imprisonment of virions could then provide an effective, curative treatment.
% % Virions invade host cells and hijack their machinery to replicate and then spread their disease. By visualising the journey of virions through the cell we may reveal finding weaknesses in its infection pathology.
% % Specifically we intend to study Herpes Simplex Virus 1, as it begins to exit a cell and spread.
% % By understanding how virions interact with cellular machinery we can provide targeted medicine
% % By visualising the journey of virions we may reveal finding weaknesses in its infection pathology
% Light sheet microscopy is the only technique capable of tracking virons which are both exceptionally small and fast

Light-sheet microscopy is at the cutting edge of live-organism imaging.
%In the coming years it will be adopted as the standard of biological imaging and transform modern biology into a highly quantitative, multidisciplinary and exciting field.
In the coming years it will help move biology out of the petri dish and back into the animal.
% Further development requires the marriage of physics, maths, statistics, computer science, chemistry and biology, in laboratories such as the Laser Analytics Group.
% I am privileged to apart of, what I believe to be, a paradigm shift in fluorescence microscopy; a field which is also highly rewarding as it marries physics, maths, statistics, computer science, chemistry and biology, a fusion I find fascinating given my physical sciences background.

%Section outlining everything


The aim of this work was to develop a \gls{light-sheet} microscope imaging system which permitted the three dimensional tracking of particles through biological samples.
%his will be used to monitor how toxic proteins travel between cells and
%how virus particles infect their host organisms with minimal %With the ability to image very quickly and tracking three dimensionally particulate targets (such as virus particles) with minimal
%photo-damage to the sample, using animal models such as drosophila.
This system was based on the work of Ernst Stelzer who pioneered digital light sheet technology\cite{Huisken2004}.
A light sheet microscope uses orthogonal illumination and detection to optically section biological samples.
A previous system was built in order to study developmental biology.
This work intended to improve upon this design so as to facilitate fast 3D particle tracking.

% The new system will:
% \begin{itemize}
% 	\item Be more vibrationally stable for low noise tracking
% 	\item Accommodate a three dimensional stage for particle tracking and volume imaging with nanometre resolution
% 	\item Use structured illumination modes that can provide higher resolution than standard illumination
% 	\item Provide more excitation wavelengths to enable improved biological flexibility in terms of fluorescent dyes that can be used with better specificity
% 	\item Feature a user-friendly software interface so that users can produce images independently with a strong programming architecture for future collaborative development.
% \end{itemize}
%The new system will be designed to be: ; ; ;  and
\pagebreak
\section{Motivation}
 %Other viruses are more deadly such as HIV, by understanding viruses medicine may be better equipped to cure or prevent infection.
%Viral infection and Alzheimer's are currently
Viruses are carriers of infectious disease in humans, by hijacking the internal working of the cell the virus replicates using the machinery of the cell.
\SI{80}{\percent} of adults in the UK are thought to be infected with Herpes Simplex Virus 1 (cold sores) which is currently medically incurable\cite{Herpes}; only the symptoms can be suppressed.
Understanding virus pathology is a requirement for assisting in therapeutic intervention.
The virus structure is well understood through high resolution techniques such as Atomic Force Microscopy and Electron Microscopy.
In this group we have used super resolution techniques to study the Herpes Simplex Virus 1 structure \textit{in vitro}\cite{Laine2015}.
Contemporary biological models of viral infectivity dynamics are based on \textit{in vitro} studies %; by tracking a single virus particle through its entire infection process \textit{in vivo}.
Studying these dynamics \textit{in vivo} and following a virus through its entire process in a living organism could provide new, useful insights and understanding which could be used to suppress or reverse viral infection in humans.
Virus particles are smaller than the diffraction limit  (\SI{20}{\nano\meter}-\SI{200}{\nano\meter}); optical super resolution techniques can image sub-diffraction limit and have observed Human Immunodeficiency Virus 1\cite{Pereira2012}.
Virus particles move tens of nanometres on the time scale of milliseconds\cite{Brandenburg2007}, these techniques currently do not produce the temporal resolution required to accurately track virus particles\cite{Brandenburg2007} in three dimensions and are limited to \textit{in vitro} studies.
%
% Dementia among the rapidly ageing first world population is becoming a heavy burden on healthcare; as of 2015 there are \SI{850000}{} people in the UK suffering with dementia\cite{Judd}.
% Alzheimer's disease (AD) is a neurodegenerative affliction accounting for 62\% of all dementia suffers.
% Amyloid fibril plaques and neurofibrillary tangles (NTF) are commonly found in post mortem AD sufferers' brains.
% It is believed that misfolded Amyloid plaques trigger the accumulation of neurofibrillary tangles and a toxic species of microtubule-associated protein, tau\cite{Ittner2011,King2002}.
% Within our group we have studied Amyloid fibril aggregation using super resolution techniques and the role of tau proteins in neuronal dysfunction.
% We have demonstrated that extracellular tau can initiate tau pathology in AD\cite{Michel2014a}, a complimentary \textit{in vivo} study on tau protein's\cite{DeCalignon2012} dynamic propagation in axons would serve to elucidate AD pathology.
%
% These issues can be addressed using light sheet technology.
% Light sheet microscopes use orthogonal plane illumination to optically section biological samples, allowing an \textit{in vivo} three dimensional study.
% Confocal microscopy also produces optical sectioning, however its raster scanning nature means it is a slow technique.
% Orthogonal illumination and detection allows detection rates comparable to wide-field.
% Light sheet technology is also a low photo-toxcity method compared to confocal and as such can image for extended periods of time at millisecond resolution.

Particle localisation techniques are compatible with light sheet microscopy and can be used to accurately localise particles to sub-pixel, sub-diffraction limited positions in two dimensions.
In conjunction with a novel third dimensional tracking technique, exclusive to light sheet, full sub-diffraction limited tracking is viable\cite{Spille2015a}.
This will then enable the \textit{in vivo} study of virus trafficking through a host cell and protein propagation in neurons with unparalleled temporal resolution.

%needed to track
%To localise sub-diffraction limited particles in a light sheet system in

%A new technique exclusive to light sheet technology will allow the tracking of sparse sub-diffraction limited particle


%Virus particles are carriers of infectious disease within humans.
%Virus structure sub-diffraction limit in size, the smallest being \SI{20}{\nano\meter} and \textit{human immunodeficiency virus type 1} being \SI{125\pm14}{\nano\meter}

%\textit{Herpes Simplex Virus 1}  being is well known from AFM and Electron microscopy.
%Recently super resolution techniques verified this structure aswell.
%A virus is composed

%Monitoring virus motion within a cell gives insights to the pathology and understanding of infectivity of the virus.

%TODO Motivations

%The motivation of this project is to aid the endeavours of biology through advanced imaging capability to tackle important biological questions.
%Questions involving diseases such as Alzheimer's and cancer so they can be better understood and thus curable


%Similiarly the

% is \SI{20}{\nano\meter}
%\begin{itemize}
%
%\item Viruses are carriers of infectious disease within humans
%
%
%\item Virus and spore structure statically well known using AFM and Electron beam.
%Dynamics, \textit{in vivo} less understood pathology currently monitored \textit{in vitro} and a full single virus nor spore particle has never been tracked through its entire infection process
%
%
%\item The applications of this microscope are varied and include:
%
%\item Virus and spore trafficking pathology in vivo.
%\item Tracking of molecule dynamics of neurodegenerative diseases such as Alzeihmers.
%\item (Maybe?) Development of cancers and tumours.
%
%\item Optical microscopy can be used to study these processes however, they are at a length of 10-200nm below the diffraction limit of visible light.
%Not only are these processes small but they are fast (cite).
%Super-resolution techniques exist to break the diffraction limit however, they sacrifice temporal resolution for spatial resolution.
%
%\item Localisation can track particles to precisions below the dffraction limit.
%
%\item Light sheet microscopy creates optical sectioning for three dimensional \textit{in vivo} study of these processes.
%A recent innovation in light sheet microscopy means that particles can be track axially as well as laterally, in real time and \textit{in vivo}.
%
%\item It is expected that this system will be able to accurately track virus and spore pathology \textit{in vivo} , a feat which has not yet been realised.
%
%\end{itemize}

%\subsection{Aims}


\subsection{Structure}

% Here, a light sheet microscope is developed to track particles in three dimensions with millisecond temporal resolution.
% Firstly the theory of fluorescence microscopy and light sheet microscopy is discussed with a comparison to other similar techniques followed by a review of particle tracking methods which are considered in the context of a light sheet microscope.
% The current biological model of virus pathology and tau protein propagation and their challenges is then presented.
% This report then discusses the methods and materials used to build a light sheet microscope up until its current state.
% Finally the progress of the microscope is summarised and the future work for the project is discussed in terms of experiments needed (once the system is operational) to determine its ability and limitations when applied to virus and tau protein tracking.
