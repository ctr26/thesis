%!TEX root = ../../thesis.tex
%!TEX enableSynctex = true
%*******************************************************************************
%****************************** Third Chapter **********************************
%*******************************************************************************
% **************************** Define Graphics Path **************************
\ifpdf
    \graphicspath{{Chapters/flopt/Figs/Raster/}{Chapters/flopt/Figs/PDF/}{Chapters/flopt/Figs/}}
\else
    \graphicspath{{Chapters/flopt/Figs/Vector/}{Chapters/flopt/Figs/}}
\fi

\chapter{Stereoscopically Reconstructed Optical Projection Tomography}

Orthogonal imaging schema can be replaced with pass through imaging provide samples are sufficiently transparent.
Instead of scanning these samples lateral one can rotate their sample and reconstruct a full three dimensional image.
This chapter addresses a key downside in traditional approaches to performing full three dimensional reconstructions tomographically.
As tomographic technology has been shrunk to the millimeter scale, errors induced by hardware become apparent.
An algorithm will be presented that relies exclusively on multiple (5>) tracked fiducially beads and reconstructs without relying on any conic fitting on said beads.
The algorithm presented will use an extension projective mathematics discussed in Chapter %TODO insert chapter.


\section{Tomography (Pedro)}
Three-dimensional imaging of anatomy in thick biological samples provides valuable data for developmental biology studies.
Tomographic techniques that generate 3D reconstructions from 2D images such as computed tomography (CT) and magnetic resonance imaging (MRI) are essential in medical applications to visualize morphology in large tissues and organs.
CT and especially micro-CT can achieve micron-scale resolution using certain contrast agents, however the high doses of radiation used make this unsuitable for repeated experiments on a biological sample.
Micro-MRI can also achieve resolution in the micron scale, however the cost and size of MRI instruments can be prohibitive for many applications[21].
Furthermore, neither of these techniques can exploit the plethora of information that can be extracted through fluorescence microscopy.

Optical Projection Tomography was first proposed by Sharpe in 2002 [30]; it uses visible light to image and create volumetric data of transparent (naturally or artificially) mesoscopic objects (1 - 10 mm) at micron-level resolution.
OPT is based on well-documented computerized tomography techniques [17] in which a set of images (we will refer to these as projections) of an object are acquired with a camera at discrete steps over a full rotation.
A cross-sectional stack of slices from the original object is reconstructed using a back-projection algorithm from the projection images.
OPT is non-invasive (although it may require fixed samples), and it can image in two different modalities: emission OPT (eOPT) and transmission OPT (tOPT).
In eOPT, a fluorescent sample is excited using a wide-field illumination source.
The photons emitted by a fluorophore of interest are collected by a detector with an appropriate filter to reject the incident illumination.
In tOPT, a broadband source with a diffuser and a collimator lens located on the optical axis (as shown in figure 1) directs quasi-collimated, uniform illumination onto the sample.
The intensity collected at the camera is a function the amount of light absorbed by the sample.
In other words, the image recorded by the detector is a projection of the attenuated radiation that traverses the sample.
These two modalities can work together to display fluorescent signals in the context of the whole organism anatomy.
OPT attempts to address the scale gap between the tomographic techniques described above (samples larger than 10 mm), and light microscopy techniques (samples smaller than 1 mm) to image biological samples in the 1-10 mm regime.

\subsection{Reconstruction}

Photons from an illumination source impinge on the sample and are attenuated as they traverse the volume of the sample towards a detector.
The detector collects a series of intensity profiles $I = I_{I}e^{-k}$ at discrete angular steps $n$ through a full rotation of the sample, where $I_{\theta}$ is the unattenuated radiation intensity from the source to the detector.
$k$ is the attenuation from all the voxels along the trajectory of a single ray, and $I$ is the measured intensity (shown in figure 3).
As mentioned in section 1, the rays from the sample to the detector approximate straight lines when their angular extent is small, so we can approximate the rays reaching the detector with line integrals.
The resulting intensity profile at the detector for a particular rotation angle is a projection, and the integral transform that results in $f(I_i,\theta_n)$ is the Radon transform.
This is defined mathematically as:
\begin{align}
    f(I,\theta) = \int \int R(x,y)\delta (xcos(\theta)+y sin(\theta)-I)dx dy
\end{align}
where $f(I,\theta)$ is the Radon transform, is the unit impulse, and $R(x,y)$  represents a 2D slice of the sample.
A parallel projection is then just the combination of line integrals $f(I)$ for a constant.% for a constant.
An inverse Radon transform is used to recover the original object from the projection data; filtered back-projection (FBP) is a standard and popular method to achieve this inversion [17].
The key behind FBP is to use the Fourier transform of each projection measurement to reorder the information from the sample onto its original place in the Fourier domain, and then take the inverse transform to recover the shape of the sample.
This is derived from the Fourier Slice Theorem [17], which states that the one-dimensional Fourier transform of a parallel projection is equivalent to a slice of the two-dimensional Fourier transform of the original object.
A filtering step is applied during back-projection to avoid spatial frequency oversampling during the object’s rotation (this
is shown in figure 5).
A high pass filter such as a ramp filter is commonly used to counter the blurring caused by this oversampling.
FBP be thought of as smearing the projection data across the image plane, and is expressed in equation form as:
\begin{align}
R_{fpb}(x,y) = \int_{0}^{\pi} f'(xcos(\theta)+ysin(\theta),\theta)dxdy
\intertext{where $f'$ is the filtered projection data, and $R_{fbp}$ is the back-projected image.}
\end{align}

\subsubsection{Aim}



\section{Stereoscopic Imaging}

\subsection{Projective geometry}

Camera imaging is governed by projective geometry
Parallel lines project onto a camera will have a vanishing points

\subsection{Camera projections}

\subsection{Multiple view scenes}
Have two views allows us to triangulate from features or fudicials in one camera to another.
Triangulation requires features in both images to be the same, this is known as the correspondence problem

Suppose we know the relative positions of the cameras and their intrinsic parameters.
Given the CCD parameters, we can translate pixel coordinates (u, v) into image plane coordinates $(x, y)$:
\begin{align}
    u = u_0 + k_u x
    v = v_0 + k_v y
\end{align}

With the focal length, image plane coordinates can be translated into a ray in 3D.
The ray may be defined by the point $\textbf{p}$ in the camera-centred coordinates where it pierces the image plane:

\subsubsection{Epipolar geoemtry}
An important part of stereo is triangulating 2 rays from a pair of image correspondences.
The most important matching constraint which can is used is the epipolar constraint, and follows directly from the fact that the rays must intersect in 3D space.
Epipolar constraints facilitate the search for correspondences, they constrain the search to a line in each image.
To derive general epipolar constraints, consider the epipolar geometry of two cameras:

The baseline is the line joining the optical centres.
An epipole is the point of intersection of the baseline with the image plane.
There are two epipoles, one for each image.
An epipolar line is a line of intersection of the epipolar plane with an image plane.
It is the image in one camera of the ray from the other camera’s optical centre to the point $X$.
For different world points $X$, the epipolar plane rotates about the baseline.
All epipolar lines intersectat the epipole.

The epipolar line constrains the search for correspondence from a region to a line.
If a point feature $\textbf{x}$ is observed in one image, then its location $\textbf{x'}$ in the other image must lie on the epipolar line.
We can derive an expression for the epipolar line.
The two camera-centered coordinate systems are related by a rotation $R$ and translation $\textbf{T}$:

\begin{align}
    \mathbf{X'}_c &= R\mathbf{X'}_c + \mathbf{T} \nonumber \\
    \intertext{Taking the vector product with T, we obtain} \nonumber \\
    T \times \mathbf{X'}_c &= T \times R\mathbf{X'}_c + \mathbf{T} \times \mathbf{T} \nonumber \\
    T \times \mathbf{X'}_c &= T \times R\mathbf{X'}_c \label{Xprime = RTX}
\end{align}

\subsubsection{Essential Matrix}

Taking the scalar product with $\mathbf{X'_c}$, we obtain:

\begin{align}
    \mathbf{X'}_c \cdot (\mathbf{T} \times \mathbf{X'}_c) &= \mathbf{X'}_c\cdot (\mathbf{T} \times R\mathbf{X'}_c)\nonumber \\
    \mathbf{X'}_c \cdot (\mathbf{T} \times R\mathbf{X'}_c) &= 0 \nonumber
\end{align}

A vector product can be expressed as a matrix multiplication:
\begin{align}
T \times \mathbf{X}_c &= T_\times \mathbf{X}_c \\
\intertext{where}
T_\times &=\begin{bmatrix}
0    & -T_z  & T_y\\
T_z  & 0     & -T_x\\
-T_y  & T_x   & 0
\end{bmatrix}
\end{align}
So equation \eqref{eq:Xprime = RTX} can be rewritten as:

\begin{align}
\mathbf{X'}_c \cdot (T_\times R\mathbf{X}_c) = 0 \nonumber\\
\mathbf{X'}_c T E \mathbf{X}_c = 0  \nonumber\\
\intertext{where}  \nonumber\\
E = T_\times R \nonumber
\end{align}

E is a $3 \ times 3$ matrix known as the \emph{essential matrix}.
The constraint also holds for rays $\mathbf{p}$, which are parallel to the camera-centered position vectors $\mathbf{X}_c$:

\begin{align}
\mathbf{p}'^T E \mathbf{p} = 0 \label{eq:pEp}
\end{align}

This is the epipolar constraint.
If we observe a point $\mathbf{p}$ in one image, then its position $\mathbf{p'}$ in the other image must lie on the line defined by \eqref{eq:pEp}.
The essential matrix can convert from pixels to rays $\mathbf{p}$ assuming a calibrated camera.
And pixel coordinates can be converted to image plane coordinates using:
\begin{align}
\begin{bmatrix}
u\\
v\\
1
\end{bmatrix}
&=
\begin{bmatrix}
k_u & 0 & u_0 \\
0 & k_v & v_0 \\
0 & 0 & 1
\end{bmatrix}
\begin{bmatrix}
x\\
y\\
1
\end{bmatrix}
\intertext{We can modify this to derive a relationship between pixel coordinates and rays:}
\begin{bmatrix}
u\\
v\\
1
\end{bmatrix}
&=
\begin{bmatrix}
\frac{k_u}{f} & 0 & \frac{u_0}{f} \\
0 & \frac{k_v}{f} & \frac{v_0}{f} \\
0 & 0 & \frac{1}{f}
\end{bmatrix}
\begin{bmatrix}
x\\
y\\
f
\end{bmatrix}
\intertext{If we define the matrix $K$ as follows:}
K &= \begin{bmatrix}
f k_u & 0 & u_0 \\
0 & f k_v & v_0 \\
0 & 0 & 1
\end{bmatrix}
\intertext{then we can write}
\mathbf{\widetilde{w}} &= K\mathbf{p}
\end{align}

\subsubsection{Fundamental Matrix}

\begin{align}
    \intertext{The epipolar constraint becomes}\\
    \mathbf{p'}^T E \mathbf{p} &= 0 \\
    \mathbf{\widetilde{w'}}^T K^{-T} E K^{-1} \mathbf{\widetilde{w}} &= 0 \\
    \mathbf{\widetilde{w'}}^T F \mathbf{\widetilde{w}} &= 0
\end{align}

F is the $3\times3$ \emph{fundamental matrix}.

%\subsubsection{Two views}
%\paragraph{Mapping from one camera to another}
%\subsubsection{Three and more views}

With intrinsically calibrated cameras, structured can be recovered by triangulation.
Firstly obtain the two projection matrices are obtained via a singular value decomposition of the essential matrix.
The SVD of the essential matrix is given by:

\begin{align}
    \hat{T}_{\times} = U \begin{bmatrix}
    0 & 1 & 0 \\
    -1 & 0 & 0 \\
    0 & 0 & 0
    \end{bmatrix} U^T
    &\text{ and }
    R = U \begin{bmatrix}
    0 & -1 & 0 \\
    1 & 0 & 0 \\
    0 & 0 & 1
    \end{bmatrix} V^T
    \intertext{Then, aligning the left camera and world coordinate
    systems:}
    P = K [I | \mathbf{0}]
    &\text{ and }
    P' = ' [R | \mathbf{T}]
\end{align}

Given the two projection matrices, we can recover structure (only up to scale, since $|\mathbf{T}|$ is unknown) using least squares.
Ambiguities in $\mathbf{T}$ and R are resolved by ensuring that visible points lie in front of the two cameras.
As with the essential matrix, the fundamental matrix can be factorised into a skew-symmetric matrix corresponding
to translation and a 3 × 3 non-singular matrix corresponding to rotation.

\section{flOPT}

\section{Results}
\section{Discussion}
Future work:
It is possible to use 3 separate views to reconstruct a scene, this involves quaternion tensors versus matrices.
The maths gets progressively harder but it is theoretically possible to input each view in directly into a special mathematical construct.
