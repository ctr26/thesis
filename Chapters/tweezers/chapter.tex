%!TEX root = ../../thesis.tex
%!TEX enableSynctex = true
%*******************************************************************************
%****************************** Third Chapter **********************************
%*******************************************************************************
% **************************** Define Graphics Path **************************
\ifpdf
    \graphicspath{{Chapter/tweezers/Figs/Raster/}{Chapter/tweezers/Figs/PDF/}{Chapter/tweezers/Figs/}}
\else
    \graphicspath{{Chapter/tweezers/Figs/Vector/}{Chapter/tweezers/Figs/}}
\fi

\chapter[Light-sheet microscopy combined with remote force measurements]{Light-sheet microscopy combined with remote force measurements}%:\\ \Large Correlating real-time viscoelastic changes with embryonic development}

%From Julien:
How multicellular organisms enact the morphogenetic programmes that ensure their characteristic forms remains an enigma.
Genetic screens have yielded an array of essential structural, patterning and signalling pathways with which morphogenesis is orchestrated.
However, morphogenesis is ultimately a physical phenomenon that requires a physical explanation.
In vivo imaging of morphogenesis allows measurements that reveal stereotypical patterns in the cellular behaviour by individual and groups of cells.
These are indicative of active force generation but are insufficient to construct a quantitative explanation of where forces are generated and how forces propagate within and between tissues.
To overcome these limitations, we require a quantitative characterisation of the physical properties of the tissues involved.
Only with this knowledge are we able to understand how forces propagate within tissues to bring about morphogenesis.
Measurements of the properties of individual cells () and for bulk tissues () have revealed both viscoelastic or visco-elasto-plastic solids ().
[Cell autonomous force generation uses actomyosin-based contractile activity. – need to mention intrinsic forces someplace]   Bulk tissue properties can be estimated using atomic force microscopy to investigate a tissue surface.
Alternatively, micropipette aspiration can probe dissociated cells or explants, however deep tissue cannot be assessed directly.
More recently, techniques have been developed to measure tissue stress and viscoelastic properties, utilising laser ablation,  oil droplets, or embedding tissue explants in matrix gel.
Most recently, ferrofluid droplets have shown that local tissue properties, that change regarding the tissue localisation.
We are still in need of methods that can give a repeated real-time readout of physical properties and relate those measurements to the underlying morphogenetic behaviour.
[**What do we say that makes us different to oil droplets?]

We sought a method that can give a non-destructive, quantitative measurement of local tissue physical properties at the length scale of a few cells, completed with seconds to minutes and repeatable over developmentally-significant periods.
We chose to use biologically-compatible paramagnetic beads, implanted into the developing zebrafish embryo
A four-pole electromagnetic device was built that produces a controlled magnetic field gradient in 3D, such that a bead can be moved with known force.
Tracking bead movement gives the dynamic material properties of the surrounding tissue.
In a first test of this approach, we have followed the emergence of the first cohesive tissue of the zebrafish blastula, between the “high” to “sphere” stages of development().
After the mid-blastula transition, mesenchymal blastomeres become first motile and adherent to form the tissue that will go on to contribute to the first morphogenetic movement of the embryo.
We could show for the first time that a three-fold elevation of tissue elasticity and viscosity are associated with this development.
This elevation is dependent upon E-cadherin-based adhesions and Rac-1 dependent cell protrusive activity, abrogating either interfered with these developmental changes.
Interestingly, reducing Rho-kinase dependent cell contractility increased both tissue viscosity and elasticity and raised the number of cell protrusions.
To better comprehend the cellular basis for the physical properties detected by our new method, we combined magnetic tweezers with light sheet microscopy.
Together, this permitted us to correlate a viscoelasticity with cell shape change and a viscosity with cell rearrangements, both cellular changes reduce as tissue elasticity and viscosity increase.
We can now assign the viscoelastic component predominantly to cell shape change and viscosity to cell rearrangement.
[Some of this really belongs in the discussion but will leave here for now]
[If stiffness is explained by cell volume, then effect should be proportional to change in r, r decreases just a few %, while stiffness changes 3x][stress stiffening]
*** how much “plastic” change is rearrangement of extracellular space? ***


\section{Tissue dynamics in developing organisms}
\subsection{Embryonic rheology}
\section{Methods of measuring tissue dynamics}
\section{Magnetic tweezers combined with Light-sheet microscopy}
\subsection{Magentic tweezer design}
\subsubsection{Maths}
\subsubsection{Mechanics}
% Fitting objectives in
% Keeping fish alive
\subsubsection{Simulations}
\subsection{Imaging Chamber Design}
%Contraints specifications and requirements
\subsubsection{Force calibration}
%Not much on this..
\subsection{Biological methods}
\subsection{Bead and cell tracking}
\subsubsection{Bead tracking}
\paragraph{Template matching}
\subparagraph{Hough-based}
\subparagraph{Cross-correlation}
\paragraph{2D Bead tracking}
\section{Results}

\subsection{Ellipsoid to sphere transformation in early embryogenesis is dominated by cell motility and migration}
After fertilisation, the zebrafish embryo undergoes series of synchronous cell divisions, followed by the first morphogenetic transformations.
Before the embryo begins its epibolic spread over the yolk cell, an initial subtle transformation occurs.
The embryo undergoes a change in shape from an ellipsoid, at high stage (3.3hpf), to a sphere at sphere stage (4hpf) (Figure 1.A-D) (Kimmel staging).
This involves both a reshaping of the blastula and its boundary with the yolk cell.
To quantify this transition, an ellipse was fitted around the projected shape of the embryo and a ratio taken of the major (animal-vegetal axis, AV, R1) to the minor (equatorial, R2) diameters.
At high stage this ratio is about 1.2 (mean = 1.18 ± 0.078 s.d.) (Figure1.E), meaning that the embryo is longer along AV axis.
During development, this aspect ratio decreases (p1k-cell/high = 0.01534, phigh-oblong = 3.585e-05, poblong/high = 1.244e-04).
By sphere stage, this ratio has reduced to ~1 (mean = 1.056 ± 0.053 s.d.), i.e. closer to spherical (phigh/sphere < 2.2e-16).

These transformations are coincident with a reduction in cell volume through cell division and a concomitant reorganisation of extracellular space. (Figure S1).
Throughout these stages, cells remain relatively loosely packed (Figure S1). Little is known about of the cellular or molecular basis of this transition.
Cells acquire motility after the mid-blastula transition (Kane \& Kimmel, 1993).
We transplanted cells expressing the actin cytoskeleton reporter Lifeact into a non-labelled recipient embryo (Figure 1.F,I), then tracked their movements and shapes in 3D.
Cells move extensively in all three directions, with no strong orientation preference (Figure 1.G,H).
Indeed, transplanted cells become dispersed in a manner reminiscent of diffusion, with a variation in mean squared displacement versus lag interval that gives an apparent diffusion coefficient of 14.6 µm2/min, (Fig s?).
We use this diffusion coefficient as a convenient measure of cell movements.
During this time, cells produce actin-enriched protrusions (Figure1.F, I).
Surprisingly, while these protrusions show no preferred orientation within the plane parallel to the embryo surface (p = 0.606) (Figure 1.J), there is a strong orientation of protrusion along the AV axis (p = 0.0053) (Figure 1.K).
Thus, the stereotypical tissue morphogenesis that shapes the embryo into a sphere is coincident with extensive cell mixing and polarised protrusive activity.
%Take from Julien's stuff
\subsection{Micro-rheology reveals increase in physical parameters during high- to-sphere- stage transformation}
Even though it has been speculated that morphogenetic change must depend upon changes in the physical properties of tissues, there is little direct evidence to support this idea.
We wish to ask if the changes that we see in cell movements and protrusive activity during the high- to sphere-stage transformation are accompanied by a modulation of physical properties of the blastoderm.
To address this question, we developed a 3D magnetic tweezer device with which we could apply a known force to a 40µm diameter paramagnetic bead implanted into the blastoderm (ref other paper).
This device creates a 3D graded magnetic field within the volume between four iron poles surrounding a suspended zebrafish embryo.
The device has been calibrated to deliver known forces along three orthogonal axes (see M\&M).
[fluorescent beads covered with poly siloxane]

We imaged implanted beads and the cells surrounding them in blastula-stage embryos for extensive periods and detected no changes in local cell arrangement, actin cytoskeleton organisation (assessed with GFP-lifeact) or myosin localisation (GFP-MII) (Figure S2).
Embryos containing a bead developed, as far as we are able to detect, unperturbed by its presence.
We measured the physical properties of the tissue by applying a calibrated, constant force (in the order of 8nN) for 1 minute (Figure 2.A) and tracking the displacement of the bead during and after application.
Force was directed alternately radially towards or away from the yolk at 3-minute intervals.
Bead trajectories reveal that embryonic tissue acts as a viscoelastic medium.
Trajectories invariably show an initial fast displacement followed by a slower, linear displacement or creep (Figure 2.B).
Upon release, the bead recoils rapidly towards its original position in a reversal of the initial fast displacement.
Displacement during the slower creep phase is not recovered (Figure 2.B).
Between high- to sphere-stage we see a significant and systematic reduction in the magnitudes of all phases of movement but not in the overall shapes of these trajectories (Figure 2.B).
To quantify and further characterise these findings, we fitted a parameterised mechanical model that accounts for the shapes of bead trajectories over time.
The most parsimonious model consists of a dashpot in series with parallel spring and dashpot (Figure 2).
The dashpots are characterized by viscous coefficients $\nu$1 and $\nu$2, and the spring by an elastic modulus, E.
Fitted parameters reveal that E increased by 3.1 times (E0min = 2.61Pa ± 0.57 s.e.m; E75min = 7.98 Pa ± 2.50 s.e.m.) and $\nu$1 by 1.8 times ($\nu$1 0min = 156.1 Pa.s ± 58.26 s.e.m.; $\nu$1 75min = 286.85  Pa.s ± 69.34 s.e.m) and $\nu$2 by 2.5 times ($\nu$2 0min = 15.82 Pa.s ± 3.20 s.e.m. ; $\nu$2 75min = 40.20 Pa.s ± 10.74 s.e.m.) over a developmental time of  ~75 min (Figure 2.C-E).
 No significant differences are found between the two directions of force application (Figure 2.C-E, pE = 0.089, p$\nu$1 = 0.427, p$\nu$2 = 0.904).
 This trend was seen irrespective of the starting developmental age of the embryo, excluding the likelihood of work hardening as an explanation.
 A developmental stiffening of the embryo was also seen in measurements of the apparent Young’s modulus as determined by AFM indentation (Figure S3).
 A characteristic time constant of the elastic deformation, $\tau$, can be derived from the ratio of $\nu$2 to E.
 Despite large changes in both of $\nu$2 and E, $\tau$ remains relatively constant over developmental time (Figure 2.F, mean = 5.33 s).
 This suggests that this tissue may contain a mechanism of self-regulation of $\tau$ or that both E and $\nu$2 are determined by a common feature.

 Our mechanical model provides a good explanation for the trajectory of the bead during active force application and the initial recoil period.
 However, bead movements in the later recovery period are more erratic and not accounted for by the model.
 We speculate that these deviations may result from additional processes, such as active cell movements.

 The fitted mechanical model implies that the tissue can be described by two mechanical elements, a soft viscoelastic component and a pure viscous component.
 We now asked if these mechanical descriptors can be understood in terms of cellular characteristics of the tissue.
 A simple hypothesis would be that the viscoelastic component derives from the mechanics of individual cells, and the purely viscosity component is a measure of cell-cell interaction.
 We address this question in two ways.
 Firstly, we visualise and measure changes in cell shapes and rearrangements within the tissue during and after bead movement; secondly, we repeat these measurements in embryos in which we have experimentally manipulated cell adhesion, cell protrusive activity and cell contractility to test their roles in determining mechanical properties.

\subsection{Elasticity is linked with cell shape deformation and Viscosity with cell rearrangement.}

To examine how cells around a bead respond during force application, we combined fast light sheet imaging with magnetic tweezers to be enable us to simultaneously track cell shapes and positions during mechanical probing (Figure 3.A-B).
We partition the time course into five periods, based upon the experimental protocol and mechanical signature described above.
We look for cellular correlates of bead displacement in the phases labelled elastic, creep and recoil.
The elastic phase is defined as 3-times the viscoelastic time constant $\tau$ (spanning 95pc of the elastic duration).
The remaining period of active bead displacement is defined as the creep phase (Figure 2.B).
Thirdly, a 3-$\tau$ period of elastic recoil is analysed as the recoil period.
Cell outlines and positions were automatically tracked and manually corrected.
Cell shape changes and cell displacements are measured along the axis of force application to the bead.
Four sectors are defined around the bead relative to this axis (Figure 3.A); we have analysed the front and rear sectors.
As a first approximation, to assess how tissue is elastically-deformed in space, we use an elastic medium force point formula (ref) that couples cell displacement inversely to distance from the centre of force application (the bead), and for a cell shape change stain rate, we use the derivative of this formula.
In front of the bead, during the elastic period, cells are both compressed and displaced forward by its movement (Figure 3.C,D).
This tendency of cells to deform diminishes over development time (Figure 3.F, plinear-regression = 0.016).
In fact, cell shape strain rates are highly correlated with and largely accounts for tissue deformation strain rate (Figure 3F, pcorrelation = 0.0001757, correlation = 0.743).
Cell shape strain rate coefficient is correlated with E (Figure 3.I, pcorrelation = 0.0449, correlation = -0.453).
Behind the bead, we can see a complementary pattern of cell and tissue stretching (Figure S4, correlation = 0.473, p = 0.0748), at all except the earliest developmental stages.
This may be due to incomplete cell adhesion, creating holes in the tissue when the bead moves with excess force.
(In these cases, cell shape change strain rates are not correlated with tissue deformation strain rate.)
When the current is turned off, the shape of cells previous compressed ahead of the bead re-expand back along the force axis (Figure 3.D,H).
Likewise, previously stretched cells behind the bead contract as the bead recoils backwards (Figure 3.D and Figure S4).
We find a good correlation between E and the rate of expansive cell shape deformation in front of the bead (Figure 3.K, correlation = 0.53, p = 0.016).
This is consistent with the rapid elastic bead recoil being determined by an elastic recoil in cell shape, after the release of the imposed force.
We conclude that the cellular signature, of the elastic periods are largely accounted for by cell shape deformations.

To investigate how tissue deforms during the creep phase, which in the mechanical model is defined by $\nu$1, we visually inspected how cells in tissue in front and behind the bead behaved during this period.
We see cells becoming compressed and moving out the plane (decreasing in area), in front of the bead and conversely cells becoming stretched and entering the plane at the rear (Figure 3 A, B).
This behaviour can be quantified by comparing changes in cell aspect ratio and area through both the elastic and creep periods (Figure 3.E).
Early cell shape changes give way to area changes during the creep period, which we interpret as the result of cells rearranging in the tissue in front and behind the bead.

To quantify cell behaviours during the creep period, we measure cell and tissue strain rates using a simplified tissue tectonics approach (REF).
Tissue strain rate, cell shape strain rate and cell rearrangement rates are measured along the direction of force application (see M\&M).
While the elastic periods were dominated by changes in cell shape, the creep period is characterised by cell rearrangements.
Cell shape strain accounts for a small fraction of tissue strain rate (Figure 3.G), and does not correlate with $\nu$1 (front correlation = 0.345, p = 0.21; rear correlation = -0.052, p = 0.85).
Nonetheless, measureable cell shape strain rates indicate some contribution persists through the creep period.
However, cell intercalation strain rate does correlate with $\nu$1, both in front (Figure 3.J, pcorrelation = 0.0638, correlation factor = 0.49) and behind the bead (Figure S4, p = 0.035, correlation = -0.546).
We conclude that cell rearrangements are the major determinants of tissue viscosity in the creep period.

\subsection{Modifications of cell motility and migration lead to defects in early embryogenesis and tissue rheology.}

The results of analysing cell shape changes and movements during bead displacement are consistent with our simple hypothesis of a two-component model.
To test this hypothesis further, we have manipulated embryos to modulate their cellular properties.
We will measure how these changes affect embryo development, cell behaviours and the mechanical properties of the blastoderm.
Cell adhesion is a fundamental requirement for building a cohesive tissue, we reduced expression of E-Cadherin (cdh1).
Using a knock-down Morpholino approach (MoECad), we show that knockdown of cell adhesion leads to embryos defective in the high-to-sphere transformation; these embryos failed to achieve a spherical shape (Figure 4.A).
At a cellular level, cell protrusions lose their radial polarity (Figure S5), while the number of protrusion per cell are not different (Figure 4B, C).
Further, cells in treated embryos have a reduced coefficient of diffusion (6.3 µm2/min, pWT-MoECad = 1.6*10-5), showing them less able to migrate as extensively as WT cells (Figure 4.B, p = 1.621e-05), though the directions of migration are as seen in WT ****[Check those directions].
Rheological measurements show that MoECad-treated embryos remain less stiff and less viscous than their WT counterparts.
Further, they are reduced in the elevation of these properties over developmental time (Figure 4.F-H).
Developmental trends are significantly different to WT, pE = 5.63e-07, p$\nu$1 = 2.10e-03, p$\nu$2 =0.042).

We have identified that cell migration and cell protrusive activity are two major cell behaviours at these stages.
Small GTPases are identified as central orchestrators of cell polarity and motility [].
We chose to manipulate two complementary molecular components of cell migration, Rac1 signalling using a dominant-negative Rac1 construct (DNRAC, ref), RhoA signalling by a dominant-negative Rho-kinase construct (DNROCK, ref).
DNRAC we anticipate should eliminate protrusive activity, while DNROCK would reduce the activation of the molecular motor myosin-2.
The injection of these dominant negative constructs affect early morphogenesis, DNRAC-injected embryos fail to progress to a spherical shape, while DNROCK injection causes a faster round-up of the embryo.
Myosin-2 inhibition via blebbistatin treatment gives the same phenotype as DNROCK treatment (Figure S5).
At a cellular level, cells in DNRAC-injected embryos produce far fewer protrusions (Figure 4B, freqWT = 2.05 extensions/frame/cell, freqDNRAC = 0.4 extension/frame/cell, p = 0.01386).
Cells in DNROCK-injected embryos make significantly more protrusions than WT (Figure 4B, freqDNROCK = 5.30 extensions/frame/cell, p = 0.0007866) and more cells display a multipolarity (Figure 4.C).
Cell protrusions in DNROCK-injected embryos favour a radial distribution but are more dispersed and numerous than WT (Figure S6).
The increase in protrusion production in DNROCK embryos is as expected for a loss of function of myosin-2 in migratory cells [ref].
(Protrusions on DNRAC injected cells are too infrequent to perform statistical analyses.)
Within DNRAC-injected embryos, the apparent diffusion of cells is significantly reduced compared with WT (4.3 µm2/min, pWT-DNRAC = 10-8).
Apparent cell diffusion in DNROCK-injected embryos is significantly increased compared with WT (29.9 µm2/min, pWT-DNROCK = 0.000433).
**** Analyses of cell migration directions in DNRAC are …., and in DNROCK appears as in WT (Figure S6).
In addition, recruitment of myosin-2 to the cell cortex is greatly reduced in DNROCK-treated cells (Figure S7).
Cortical myosin2 resembles the WT distribution in both DNRAC and MoECad injected embryos (Figure S7).
Interestingly, at both cellular and embryonic levels, the phenotypes of DNRAC and DNROCK treated embryos differ from WT in opposite ways, including embryo morphology, cell protrusion formation, cell migration and cell dispersion.
In the light of the MoECad results, we would predict that the physical properties of DNRAC-injected embryos should be less stiff and less viscous than WT those for DNROCK-injected would be stiffer and more viscous.
When we measure those rheological parameters by magnetic bead displacement, we confirm these predictions: DNRAC embryos fail to increase stiffness and viscosities $\nu$1 and $\nu$2 (pE = 1.016e-09, p$\nu$1 = 1.87e-04, p$\nu$2 = 0.0028) while DNROCK embryos become stiffer and more viscous (pE = 7.59e-25, p$\nu$1 = 6.74e-15, p$\nu$2 =3.65e-12) than their WT counterparts (Figure 4.F-H).
Interestingly, $\tau$ is affected by the different knock downs, stiffer embryo showing a shorter $\tau$ than softer ones (all p << 1e-5).

Finally, we wished to ask how the changes rheological properties found for DNRAC and DNROCK treatments are reflected in the rates of cell and tissue deformation?
We repeated fast light sheet imaging analyses with each knock-down treatment, to follow cell behaviours during force application.
The resultant parameters do segregate according to their mechanical properties: softer DNRAC embryos present more cell deformation during the elastic period (pWT-DNRAC = 0.0013) and more cell rearrangement during the creep period (pWT-DNRAC = 0.0025), compared to WT.
Cells in the stiffer DNROCK-injected embryos deform less in the elastic period (pWT-DNROCK = 0.017) and induce less cell rearrangement in the creep period (pWT-DNROCK = 0.13) (Figure 4.J).
We can now combine data from all WT and experimental manipulation conditions to examine correlations within an extended range of developmental conditions.
We confirm that the previously-observed correlations between cellular and rheological parameters hold true under this expanded range.
Stiffness E is strongly correlated with initial cell shape deformation while $\nu$1 is strongly correlated with cell rearrangements.
This suggests that there may be a relatively simple cellular interpretation of the physical model measured using magnetic bead rheology.
Since the measurement of bead movements are faster, easier and more sensitive than the associated cell measurements, this facilitate us gaining greater insight into the mechanics of developing tissues.
\section{Discussion}
%New technique for remote measurement of biological forces
In this study, we have shown for the first time the direct link between cell behaviours and embryo morphogenesis through their measured mechanical properties, in the of the zebrafish blastula simple morphogenetic transformation.
We could apply directed force within a tissue and challenge this tissue mechanically and report the response of the tissue at cellular level.
We also showed that these properties are controlled at molecular level, through a network made of actomyosin and cadherin as knock downs of Rac1, Rho-kinase and E-Cadherin have dramatic repercussions on all levels, from cell to embryo.


\subsubsection{Rear versus Front}
It is still not clear what happens at the rear of the bead, especially at early stages.
The bead is not coated with any molecules.
The way the bead is contacting surrounding cells is thus dependent on how cells are adhesive between each other.
When this adhesion is weak, at early stages in WT, or in soft embryos, gaps are observed at the rear.
Doing so, the cells are not pulled by the bead movement.
When the connectivity becomes stronger, we can clearly a stretch of the cells at the rear that is proportional to E.

\subsubsection{Recovery phase}
After ~45sec sec after cutting the current off, the rheological model lack to explain the bead movement.
What could possibly happen is that we enter a phase during which morphogenetic movement is overtaking bead movement as cells are active.

\subsubsection{Is myosin-2 activity acting unexpectedly?}
Inhibition of cell contractility gives at rheological level unexpected results: absence of myosin-2 activity gives rise to stiffer embryo.
Canonically, myosin controls cell contractility and loss of contractility leads to softer cell cortex.
Nonetheless, these expectations are mostly rising from epithelium and single cell studies.
Here, the tissue is strictly mesenchymal and cells present a strong protrusive activity.
In fact, at cellular levels, it is known that regulation of actomyosin mechanics is dependent on the nature of the cell, as well as how cells are organised.
For instance, myosin-2 is controlling cell cortex stiffness, but in two different ways: for cells in suspension, myosin-2 softens the cortex, while, if cells are plated on a dish, myosin-2 stiffens it.
This paradox behaviour is also noticed at molecular levels, when myosin activity can soften or stiffen the actin network, regarding what molecular actors are present.
Expectations about myosin incidence on cell and tissue stiffness must take in account the cell organisation and how cells connect to each other.

\subsubsection{Connectivity: rising of tissue properties}
Interestingly, we demonstrate that the emergence of mechanical properties is linked with the motility of the cell.
In particular, we showed that dynamics of protrusive activity and cell-cell adhesion are two main actors.
This suggests that the establishment of connection between the cells is a key factor and control of coherent tissue.
This connectivity can be seen as a superstructure of the actomyosin network connected between cells through cell adhesion molecules, such as E-Cadherin.
This network would be dependent upon how cells contact each other through cell extensions, a turnover between stable contacts and new contacts, and, upon how this network grows and contract.
[Maybe do here a hypothesis about syntitial embryos of Drosophila]

\subsubsection{Correlation of rheological parameters}
Intriguingly, we were not able to uncouple the different mechanical parameters in the three different knockdowns we looked at.
Parameters variate in the same direction.
It is still unclear the fundamental mechanisms by which a particular type of molecules could influence all parameters.
This suggests strongly that tissue rheology is regulated at higher level than simply molecular and cellular.
It shows that even if cells and molecular network have their own mechanical specificities, they must be integrated in a greater framework to predict their properties.
