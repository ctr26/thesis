%!TEX root = ../../thesis.tex
%!TEX enableSynctex = true
%*******************************************************************************
%****************************** Chapter **********************************
%*******************************************************************************
% **************************** Define Graphics Path **************************
\ifpdf
    \graphicspath{{Chapters/conclusions/Figs/Raster/}{Chapters/conclusions/Figs/PDF/}{Chapters/conclusions/Figs/}}
\else
    \graphicspath{{Chapters/conclusions/Figs/Vector/}{Chapters/conclusions/Figs/}}
\fi

\chapter{Discussion and Conclusions}
\chapter{Perspectives}
Exotic Illumination
  Mention Daisy's work, show design?

  Confocal slit scanning was found to complement Bessel beams by rejecting light generated from higher orders of the beam (Florian O Fahrbach et al. 2013b).

    They were combined using dichroic mirrors and then expanded twice to match the size of a liquid crystal SLM (Hamamatsu).
    When inactive, the SLM reflected the beam into a beam dump.
    In normal operation a phase gradient was applied on the SLM to reflect the beam into the microscope.
    This arrangement ensured that the higher orders of the beam generated by SLM pixilation were directed to the beam dump.
    The SLM was positioned such that it was in a back focal plane of a long focal length (400 mm) lens, while its focus coincided with the first galvanometric scanner.
    A flip mirror was positioned between first lens in the first beam expander and the 400 mm lens.
    Thus it was possible to bypass the SLM if necessary (e.g. during assembly and testing and when SLM was not operational).
    Another flip mirror was added after the 400 mm lens to couple in the original laser sources through tuneable lenses.
     The excitation path after the galvanometric scanner remained unchanged.

  A time-gated FLIM detection system is available in the Chemical Engineering Department and could be added to the LSFM system.
  The FLIM detection was a single unit containing a low NA beam expander, a time gated image intensifier (LAVision) and a high NA beam expander relaying the intensifier on a camera.
  In the new design it replaced the camera for FLIM imaging.
