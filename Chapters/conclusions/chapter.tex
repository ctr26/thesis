%!TEX root = ../../thesis.tex
%!TEX enableSynctex = true
%*******************************************************************************
%****************************** Chapter **********************************
%*******************************************************************************
% **************************** Define Graphics Path **************************
\ifpdf
    \graphicspath{{Chapters/conclusions/Figs/Raster/}{Chapters/conclusions/Figs/PDF/}{Chapters/conclusions/Figs/}}
\else
    \graphicspath{{Chapters/conclusions/Figs/Vector/}{Chapters/conclusions/Figs/}}
\fi

\chapter{Discussion and Conclusions}\label{chapter:conclusions}

\epigraph{\emph{You do the light-sheet, we do the heavy sheet}}{--- Romane Lane}

\chapter{Perspectives}
Exotic Illumination
  Mention Daisy's work, show design?

  Confocal slit scanning was found to complement Bessel beams by rejecting light generated from higher orders of the beam (Florian O Fahrbach et al. 2013b).

    They were combined using dichroic mirrors and then expanded twice to match the size of a liquid crystal SLM (Hamamatsu).
    When inactive, the SLM reflected the beam into a beam dump.
    In normal operation a phase gradient was applied on the SLM to reflect the beam into the microscope.
    This arrangement ensured that the higher orders of the beam generated by SLM pixilation were directed to the beam dump.
    The SLM was positioned such that it was in a back focal plane of a long focal length (400 mm) lens, while its focus coincided with the first galvanometric scanner.
    A flip mirror was positioned between first lens in the first beam expander and the 400 mm lens.
    Thus it was possible to bypass the SLM if necessary (e.g. during assembly and testing and when SLM was not operational).
    Another flip mirror was added after the 400 mm lens to couple in the original laser sources through tuneable lenses.
     The excitation path after the galvanometric scanner remained unchanged.

  A time-gated FLIM detection system is available in the Chemical Engineering Department and could be added to the LSFM system.
  The FLIM detection was a single unit containing a low NA beam expander, a time gated image intensifier (LAVision) and a high NA beam expander relaying the intensifier on a camera.
  In the new design it replaced the camera for FLIM imaging.


  TODO \subsection{Expansion Microscopy}

  \section{Clearing}

  Optical-clearing optimization of LSFM
  Though not specific to LSFM, refractive-index matching by chemical clearing of tissue finds a natural home in this context, which allows for exceptionally large, fixed samples to be imaged with microscopic resolution and in a reasonable period of time.
  Nevertheless, the transition to larger sample sizes does provide some unique optical challenges, and although clearing makes even deep tissues accessible, without corresponding changes to the optical components they remain tantalizingly out of reach.
  % Dodt et al. reported an ultramicroscope that uses low magnification and NA optics to image cleared mouse brains over centimeter-sized FOVs63 naturally, with some sacrifice to spatial resolution.
  The generation of thinner, less-divergent light sheets benefits sub-cellular and macroscopic LSFM imaging alike.
  Saghafi et al. were able to shape the illuminating light sheet using several aspheric and cylindrical lenses in series to produce light sheets with a 4-μm thickness at the beam waist and with little divergence over several millimeters64.
  Others have used binary-pupil phase masks to achieve similar results65,66.
  Tomer et al. adopted a different approach to imaging optically cleared tissues in the CLARITY optimized light-sheet microscope (COLM)67, which tiles the acquisition process to cover large FOVs.
  The superior collection efficiency afforded by high-NA optics compensates somewhat for the additional exposures by making better use of the available light, whereas the relatively high magnification and NA affords submicron resolution.
  To compensate for misalignment caused by residual refractive index inhomogeneities deep inside tissue, an autocalibration routine adjusts the light-sheet position such that the two planes maintain co-alignment throughout the volume.
