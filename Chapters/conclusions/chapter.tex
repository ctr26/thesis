%!TEX root = ../../thesis.tex
%!TEX enableSynctex = true
%*******************************************************************************
%****************************** Chapter **********************************
%*******************************************************************************
% **************************** Define Graphics Path **************************
\ifpdf
    \graphicspath{{Chapters/conclusions/Figs/Raster/}{Chapters/conclusions/Figs/PDF/}{Chapters/conclusions/Figs/}}
\else
    \graphicspath{{Chapters/conclusions/Figs/Vector/}{Chapters/conclusions/Figs/}}
\fi

\chapter{Conclusions}\label{chapter:conclusions}

%INTO LIT AND DESIGN
In this work an \gls{light-sheet} fluorescence microscope was designed built and optimised for organismal and cellular imaging.

\section{Microscope design and construction}

The microscope presented in \ref{chapter:design} was able to record imaging volumes at a rate of \SI{512x512x100}{\micro\meter} volumes per second, at maximum running speed.
It featured four laser channels spanning the visible spectrum and suitable for standard fluorescent dyes.
The microscope generated light-sheets digitally by sweeping a galvanometric mirror converting a standard Gaussian beam into a Gaussian sheet.
The optical elements on the \gls{illumination board} consisted of a beam splitter for creating two beams and an optical \gls{SLM} path for generating exotic illumination or two beams directly, steering mirrors and relay lenses.
A XYZ stage was used to mount and position samples with sub-\SI{}{\micro\meter} precision.
The imaging plane of the  detection objective was matched to the scanning light-sheet by using a fast piezo electric objective actuator.
An additional objective lens relay was positioned in front of the detection camera to increase the total magnification from \SI{25}{\times} to \SI{31.25}{\times} or \SI{62.5}{\times} when using the \SI{1.25}{\times} and \SI{2.5}{\times} objective lenses respectively.
This provided a variable field of view for imaging organisms and single cells, with the largest magnification being oversampled compared to Nyquist's limit.

%TODO desgin description
%Homography
\section{Homographically generated light-sheets}
In Chapter~\ref{chapter:homography} the method by which the generation of voltage waveforms was improved upon.
Standard methods of generation rely on three positional coordinates being record for registering the illumination volume to the imaging volume.
To use the four available coordinates additional non-linear mathematics is needed to project affine input coordinates onto the resultant non-affine illumination space.
The non-linearity observed (figure) arises from imperfections in the scanning optics, this was characterised for the scan lens which converts beam angle into beam position at the sample volume.
When comparing the 3~point and 4~point correction it was found that there was an average \SI{42}{\percent} increase in recovered signal from fluorescent beads.

%Flopt
\section{Frame localisation optical projection tomography}
Chapter~\ref{chapter:flopt} then used the mathematics presented in Chapter~\ref{chapter:homography} to address the issue of computer-tomography reconstructions in systems prone to mechanical error; in this case for optical projection tomography.
The ubiquitous Radon transform is typically used to reconstruct projected data sets but take no account of motions beyond rotation about the axis normal to the plane projected.
This assumption is valid for stable systems, but leads to reconstruction error for systems with limited available angles as well as systems with mechanical jitter and systematic drift.
Provided there are 6 or more fiducial markers common to each adjacent image pair from an \gls{OPT} image set, each point can be triangulated and the resultant rotation and translation matrices between each image pair can be computed.
Each image from the set can then be computational back-projected and aligned using the recovered matrices.
The resultant volume is then filtered to provide a faithful reconstruction of the volume that is robust to systematic mechanical errors as well as jitter.

%DUAL SLIT
\section{Confocal \gls{slit-scanning} microscopy}
The illumination and imaging volume registration was also critical for Chapter~\ref{chapter:dualslit} where synchronisation of the rolling shutter on the camera and the sweeping of the beam had to be well matched.
By synchronising the rolling shutter and the beam sweeping images can be recorded with better optical sectioning and higher contrast as out-of-focus light is rejected by the virtual pinhole in the direction of the rolling shutter.
This was implemented on the light-sheet microscope and was characterised to show the maximal \gls{SNR} was when the shutter width and the beam waist were comparable in size.
The firmware on the camera was then upgraded along with the beam creation optics such that two shutters would roll synchronously and two beams would be generated to match them.
This then increased the achievable frame-rate of the slit-scanning camera from \SI{50}{\hertz} to \SI{100}{\hertz}, the maximum of the camera.
This was in preparation for the imaging capabilities needed for tracking virus particles as in Chapter~\ref{chapter:spt}.
The effect of slit-scanning was also simulated for light-sheet systems as well as wide-field systems.
This included structured illumination systems where it was shown that a marked rate of imaging and contrast improvement could be obtained.

%Chamber
\section{Open-hardware sample mounting}
In preparation for the two chapters addressing biological questions a sample chamber was designed and presented in Chapter~\ref{chapter:chamber}.
The chamber was a \gls{3D} printed design capable of mounting cells  and organisms, with fixed cells and live organisms being shown in this work.
\gls{3D} printed materials were used for the rapid prototyping capability as well as the advantage of producing multiple copies quickly when compared to metal.
It is possible to sterilise \gls{3D} though the more aggressive methods were damaging to the printed chambers.

The key novelty of the chamber was the use a shelf for sample mounting.
This allowed for the mounting of flat glass coverslips, which are viable for use in common cell and organism mounting procedures.
It also allowed for the use of a large opening angle detection objective lens mounted at \SI{45}{\degree} in conjunction with a high NA illumination objective lens.
Enabling the use of these two objectives allowed for the imaging of organisms as well as sub-cellular activities in cell culture.
The chamber include addition design features such as: chamfered edges in the medium containing section to act as hard stops if protecting the objective lenses and minimise the use of media; a positing camera and window from with a couple infrared and white light source for transmitted light imaging; a side window for positing samples by eye; a drip-tray for instrument safety against spillages; module for inserting heating elements and magnets to be used for securing samples and cleared through-holes for bolting the chamber to a metric breadboard.

\section{Particle tracking for viral egress}
With such a chamber now available, imaging \emph{in vivo} sub-cellular processes was now possible, see Chapter~\ref{chapter:spt}.
To then monitor viral egress, the light-sheet microscope required modifying such that it could axially localise single particles within the light-sheet.
From there the light-sheet would be repositioned and the process iterated for each recorded frame giving a sub-diffraction limited axial position over the viable \SI{100}{\micro\meter}.
As such, a weakly cylindrical lens was inserted into the imaging path and positioned to provide a suitable amount of astigmatism for the available pixels per (\SI{4x4}{}) \gls{PSF} at 62.5 (\(25\times2.5\) magnification.
It was shown that the calibration for a recorded astigmatic \gls{PSF} was suitably linear for the purposes of calibrated tracking when using template correlation or template covariance.

Dynamic blur for an astigmatic \gls{PSF} was then considered.
By using a Gaussian model of an astigmatic \gls{PSF} it was shown that, using typical parameters as found for the light-sheet microscope here, there is an average \SI{25}{\percent} error axial localisation.
The magnitude of this error could potentially explain why single particles do escape tracking before they bleach as demonstrated by Spille~\emph{et.~al}~\cite{spilleDirectObservationMobility2015}.

Cells infected with \gls{HSV}-1 were then fixed and mounted on glass cover slips for imaging.
Three viral proteins were labelled to demonstrate the feasible of tracking single virus particles in a live cell.

\section{Light-sheet microscopy and magnetic tweezers}
Finally, in Chapter~\ref{chapter:tweezers} a magnetic tweezer system was developed and used to remotely measure the viscoelastic properties of a developing zebrafish embryo.
A simple mechanical model of the response of the embryo was made, consisting of dash-pots and springs, and was fitted to the trajectory of a tracked bead from push-pull experiments.
Viscoelastic properties were then extracted from the model fitting and then monitored over \SIrange{1}{2}{\hour}.
This was the first demonstration of a direct link between cell behaviours and embryo morphogenesis arising from changing mechanical properties.
The blastula tissue was shown to have an intrinsic viscoelastic response.
The stiffening blastula, composed of radially anisotropic cells, may drive blastula thinning and yolk bulging in the next morphogenetic movement of the zebrafish embryo.
Interestingly, the viscoelastic timescale parameter, \(\tau\), remained constant across both time and embryonic mutations.
%TODO cleanup
%%
% The study presented here is the first demonstration of a direct link  between cell behaviours and embryo morphogenesis arising from changing mechanical properties.
%By applying a directed local force to a developing embryo, through an embedded magnetic bead, local and ensemble mechanical properties of tissue were characterised.
%The blastula tissue was shown to have an intrinsic viscoelastic response; the data showed that tissues exhibit fluid-like behaviours, as observed in previous studies using optical tweezers in mature epithelia, [?] and magnetic droplets injected into much older fish than considered here [?].
%The viscoelastic timescale parameter, t, remained constant across both time and embryonic mutations.
%The results of this study elucidate the biological meaning of the rheological parameters of viscosity and elasticity in terms of cellular morphology changes and rearrangement.
%
% The viscous and elastic components are involved in cell shape, intercalation and deformation. Increased stiffness is related to less deformable cells, and increased viscosity to diffusivity of cells during rearrangement. In developing wildtype embryos, prior to bulging, the blastula becomes 3-fold more viscous and stiff.
% Tissue viscosity, in the form of cell rearrangement, is a measure of the friction imposed by cell-cell connections, which may be mediated by cell protrusions and
% E-cadherin adhesion. Rac1 activity would then follow to promote viscosity and cell stiffness by increasing the polymerisation of actin and encouraging cell-cell adhesions via E-cadherin connections.
%Yet, the decrease in cell stiffness with MoECad treatment suggests a more complicated coupling [of what and what? Ecad and Rac1?].
%We propose that a cellular mechanotransduction mechanism assays the compliance of the environment between sites of adhesion.
% It is well understood that cells grown on a surface adjust their mechanical stiffness in proportion to substrate stiffness [? ? ].
%In the case of early mesenchymal tissues, we suggest that the cells themselves collectively constitute their own local mechanical environment, coupled via protrusion-promoted cell-cell adhesion.
%Numerous examples exist of mesenchymal cells interacting via protrusions [ref].
%During collective migration, cadherin-mediated cell-cell connections are able to promote cell polarisation and cytoskeletal rearrangements, acting through Rac1 activation [? ].
%Given the role of protrusions in both motility and transduction, this relationship explains the close correlation between viscosity and stiffness.
% The emergence of cohesive mesenchymal tissue mechanical properties may be linked with the motility of its cells and, in particular, the formation of protrusions that facilitate the establishment of cadherin-based cell-cell adhesions.
%Connectivity could result in an actin superstructure networked between cells, through cell adhesion proteins.
%Tissue stiffness and viscosity are well-correlated with the number of protrusions present and it is possible that these mechanisms might be found in other mesenchymal tissues during development [can you back this up w/ref?].
%These findings highlight a potentially fundamental difference in the determination of mechanical properties between epithelial and mesenchymal tissues.
%Epithelial cells represent relatively static configurations, based upon adhesion and contraction, while mesenchymal cells are adhesive and stiff, but dependent upon dynamic protrusion-based motility.
%We speculate that the stiffening blastula, composed of radially anisotropic cells, is thus poised to drive blastula thinning and yolk bulging in the next morphogenetic movement of the zebrafish embryo.

\chapter{Perspectives}\label{chapter:perspectives}

Looking forward there are several key areas with which this body of work could be extended and improved upon.

\section{Light-sheet microscope developments}
\subsection{Illumination}

% Currently the system features a good range of visible spectrum wavelengths for excitation; however t
The system would benefit from having thinner light-sheets, particularly for single cellular imaging where axial resolution is more important when compared with larger organism imaging.
Exotic illumination modes were explored briefly during this work and a Bessel beam was constructed using the \gls{SLM} as featured in Chapter~\ref{chapter:dualslit}.
As discussed Bessel beams can be used to create thinner light-sheets over a range of \SI{\sim40}{\micro\meter}.
Similarly, Airy beam light-sheets could be created using a cubic phase mask on the \gls{SLM}.
% are also feasible in this system which
Airy Beams behave over a longer range (\SI{500}{\micro\meter}) compared to Bessel beams though they require deconvolution in the final step which can take a long time with large data sets.
Alternatively a \gls{2P} beam could also be used to create a light-sheet thinner than single photon illumination, though infrared lasers are expensive and would require new optics.

\subsection{Imaging}

During this work a plug-and-play \gls{FLIM} enabled camera became commercially available for wide-field microscopes, combined with the presented light-sheet microscope this could provide a fast volumetric fluorescence lifetime readout from cells to organisms.
Stelzer~\emph{et. al} showed that volumetric light-sheet combined with FLIM can be useful in organismal imaging~\cite{gregerThreedimensionalFluorescenceLifetime2011a}.
FLIM combined with the viscoelastic measurements seen in the Chapter~\ref{chapter:tweezers} could further elucidate tensions occurring locally to the bead under motion, with a suitable fluorescent reporter~\cite{colomFluorescentMembraneTension2018}.
Fluorescence lifetime could also be used for indicating if a virion has moved between boundaries within a cell or collected tegument proteins, though the \gls{SNR} of single particle lifetime measurement may be too low.

\section{Waveform generation}

For waveform generation a 4~pt calibration was presented which gave a \SI{42}{\percent} improvement in signal collecting efficiency, by matching the illumination and imaging volumes better than a 3~pt calibration.
It is possible to create a 5 pt calibration using elastic transforms (elastic transforms are needed as the projective mathematics in Chapter~\ref{chapter:homography} breaks down beyond 4~pts), which could further increase the collection efficiency.
With calibrations beyond 4~pts it becomes difficult to find the corresponding point between the beam and the image.
The most obvious next point to choose beyond the 4 available corners of the illumination volume, would be the centre point of the those corners at \(z:\SI{100}{\micro\meter},x:\SI{1024}{\text{px}}\).
Finding the focus at this \(z\) position would involve maximising the image maximum when imaging in dye (as in Chapter~\ref{chapter:homography}), matching the \nth{1024} pixel with the middle of the beam would be more difficult, particularly by eye.
As a result, it would best to position the 5th point calibration automatically by using a peak finding algorithm for the axial (\(z\)) focus matching, using image maxima; and in (\(x\)), by maximising the summed intensity along the requisite pixel row.
In doing so, all other available pixel rows in \(x\) would be trivial to calibrate to, as would all axial positions.
From this, a complete look-up table of positions could be created for the scanning beam.
Such an approach would mean each imaging session would require a potentially lengthy calibration, but for the benefit of not needing to generate waveforms during imaging and having very precisely focused light-sheets.

\section{Slit-scanning}

\subsection{Light-sheet slit scanning}

Confocal slit-scanning was found to complement Bessel beam illumination by rejecting light generated from the higher orders of the beam~\cite{fahrbach_rapid_2013}.
By using the dual-beam optics as presented in Chapter~\ref{chapter:dualslit}, two Bessel beams could be created to provide \SI{100}{\hertz} Bessel sheet imaging for high axial resolution sub-cellular imaging.

\subsection{Wide-field slit-scanning}

The slit-scanning theory presented in Chapter~\ref{chapter:dualslit} for non-light-sheet microscopes will need to be implemented on a real system to validate the simulations presented in Chapter~\ref{chapter:dualslit}.
The National Physical Laboratory has such a \gls{SIM} available for use, with a sufficiently fast \gls{SLM} to match the speed of the rolling shutter with a scanning illumination line.

\subsection{Slit-scanning fusion imaging}
A further extension of this work could involve using two cameras and an image splitter.
By rotating the additional camera at the second port of the image splitter and superimposing two orthogonal sweeping lines on the \gls{SLM}, two images of the same sample would be produced with increased contrast and resolution in orthogonal directions.
A joint Richardson-Lucy deconvolution~\cite{ingaramoRichardsonLucyDeconvolution2014} could then be used to fuse the to images into a homogenous high resolution and high contrast image.
As an image splitter would be used, a brighter image would be needed to maintain the \gls{SNR}, this could be achieved using more illumination light or increasing the overall exposure time.
The method proposed would be able to achieve potential super-resolution at video rates.

\section{Frame localisation optial projection tomography}
% flopt - attempt reconsutrction with real data
% Fiducial free provided there is sufficient texture

Chapter~\ref{chapter:flopt} presented a theoretical framework for the reconstruction of \gls{OPT} data sets that was robust to systematic mechanical drift and noise.
To verify that the reconstruction works beyond simulated data, an \gls{OPT} image set with 5 or more fidicual beads will need to be reconstructed.
Watson~\emph{et al.}~\cite{watsonOPTiMOpticalProjection2017}, have published a set of mechanical adaptors and softwares so that such a data set could be made using a standard inverted fluorescence microscope.
Given an appropriate dataset, a particle tracking algorithm for the labelling trajectories that is robust to occlusions will also be needed.

\subsection{Fiducial-free optical projection tomography}

In terms of developments to the theory presented in Chapter~\ref{chapter:flopt}, it may be possible to produce a fiducial-free reconstruction provided the sample has sufficient visual texture.
Texture here refers to variance of contrast within the image, an example would be the transmitted white light image of a developed Zebrafish.
In computer vision, the fundamental, essential and homography matrices are found from image pairs with no placed fiducial markers.
Instead, features that are common to the each view are used as points for triangulation.
The confidence in such points is much lower than the defined manually placed fiducial marks, but, many more can be used, increasing the overall confidence in the recovered matrices.
From here, the recovered matrices can be deconstructed and deployed as described in Chapter~\ref{chapter:flopt}.
This could be implemented provided a \gls{tOPT} dataset were available.

\section{The sample chamber}

The current sample chamber design as shown in Chapter~\ref{chapter:chamber} does not provide atmospheric control beyond closed-loop temperature control, which is needed for some organisms and \ce{CO_2} buffered media.
To make the chamber viable for long-term live cellular imaging, a suitable lid would need to be designed and added to the chamber.
The top of the sample chamber was purposefully designed flat so that such a lid could be created and attached.
The intention was to use the standard chamber to position a sample and then attach the appropriate lid for long term imaging.
The lid would need an entry point for gas flow and the lid would be supported by the walls of the chamber below.
Gaskets would be fitted around the objectives to reduce gas loss and an O-ring would couple the lid and sample chamber to a mostly air-tight fit, though some flow would be needed to not pressurise the vessel.

\subsection{User construction}

Ideally the chamber would be printed with the option to have clear windows without the need for user intervention i.e.\ glueing perspex windows in manually.
However, advances in the \gls{3D} printing technology would be needed before the chamber could be printed from a single piece.
This would require optically clear printing which, using extrusions techniques currently, is impossible.
It may be possible to \gls{3D} a design for casting a sample chamber in one piece from a curable plastic, however this method would be much more involved than the method proposed.

\section{Viral imaging using single particle tracking}

It was shown in Chapter~\ref{chapter:spt} that the light-sheet microscope was capable of localising a static single particle.
However, efforts are needed to demonstrate this in dynamic systems.
One approach would be tracking a single particle in a known viscosity liquid as discussed in Chapter~\ref{chapter:spt}.
Another approach could involve keeping the light-sheet static and measuring the ensemble statistics of many particles being tracked.
The latter approach would, however, lead to a bias towards short tracks of particle trajectories.

\subsection{Viral imaging in live cells}

Virion movement would need to be demonstrated in live cells to show that cells survive and behave as expected in the light-sheet microscope; and that virions can be tracked when moving within a cell, at speed.

\section{Magnetic tweezers and light-sheet microscopy}

The combined technique of remote force measurement using magnets and light-sheet microscopy is now being developed into a stand-alone device in the Cambridge University Engineering Department by Fergus Riche.
The device will continue to be applied to observing the mechano-biology of zebrafish as well as the characteristics of tissues. %bio-films and cells.
The new device is based around a flat light-sheet design, with a chamber constructed around the objectives permanently embedded with quadrupole magnetic tweezers.

\section{Additional samples}

Mammalian cells and Zebrafish were the specimens of choice for this work as they are both optically clear and biological relevant models.
Human and mouse embryos were also considered as they are more biologically relevant for the study of humans and mammals.
Issues with mouse embryos are that of collection yield, with the required sacrifice of a mouse to produce a mouse embryo, as well as the atmospheric conditions required to image the embryo.
Human embryos are also much less abundant than Zebrafish embryos as well as requiring a very specific environment to survive.
That said, given the capabilities of the chamber developed in Chapter~\ref{chapter:chamber}, both options could be explored as both embryo types would benefit from the fast and gentle volumetric imaging provided by light-sheet microscopes.

Organoids, which are miniaturised forms of organs built or grown from organ samples, would also benefit from light-sheet imaging.
The issue for imaging organoids lies with them being often very opaque to visible light.
Recent advances in clearing techniques and expansion microscopy would make the imaging of large fixed organoids (including brain organoids) viable for light-sheet microscopy.
Clearing techniques use chemical regents to wash away scattering material from samples, particularly highly scattering lipids, to reduce the opacity of the sample~\cite{_clearing_2016}. %TODO Check
Similarly expansion microscopy washes away all but the fluorophores and a gel scaffold leaving a fluorescent ghost of a sample.
Water is then added to linearly swell the sample to many times its original size for super-resolved imaging for diffraction limited microscopes~\cite{chen_expansion_2015}.
Light-sheet microscopy would be particularly valuable for expanded samples as the amount of time required to image an expanded Drosophila brain, for example, can be on the order of months using laser scanning techniques.
Both clearing and expansion techniques suffer from sample fragility and collection yield due to the harshness of their respective processes.

%
% % samples - expansion microscopy, clearing, mouse/human embryos
%
% TODO \subsection{Expansion Microscopy}
%
% \section{Clearing}
% %
% % Optical-clearing optimization of LSFM
% % Though not specific to LSFM, refractive-index matching by chemical clearing of tissue finds a natural home in this context, which allows for exceptionally large, fixed samples to be imaged with microscopic resolution and in a reasonable period of time.
% % Nevertheless, the transition to larger sample sizes does provide some unique optical challenges, and although clearing makes even deep tissues accessible, without corresponding changes to the optical components they remain tantalizingly out of reach.
% % % Dodt et al. reported an ultramicroscope that uses low magnification and NA optics to image cleared mouse brains over centimeter-sized FOVs63 naturally, with some sacrifice to spatial resolution.
% % The generation of thinner, less-divergent light sheets benefits sub-cellular and macroscopic LSFM imaging alike.
% % Saghafi et al. were able to shape the illuminating light sheet using several aspheric and cylindrical lenses in series to produce light sheets with a 4-um thickness at the beam waist and with little divergence over several millimeters64.
% % Others have used binary-pupil phase masks to achieve similar results65,66.
% % Tomer et al. adopted a different approach to imaging optically cleared tissues in the CLARITY optimized light-sheet microscope (COLM)67, which tiles the acquisition process to cover large FOVs.
% % The superior collection efficiency afforded by high-NA optics compensates somewhat for the additional exposures by making better use of the available light, whereas the relatively high magnification and NA affords submicron resolution.
% % To compensate for misalignment caused by residual refractive index inhomogeneities deep inside tissue, an autocalibration routine adjusts the light-sheet position such that the two planes maintain co-alignment throughout the volume.


%
% \chapter{Perspectives}
%
% %DESIGN
%
% Exotic Illumination
% Mention Daisy's work, show design?
%
% Confocal slit scanning was found to complement Bessel beams by rejecting light generated from higher orders of the beam (Florian O Fahrbach et al. 2013b).
%
% They were combined using dichroic mirrors and then expanded twice to match the size of a liquid crystal SLM (Hamamatsu).
% When inactive, the SLM reflected the beam into a beam dump.
% In normal operation a phase gradient was applied on the SLM to reflect the beam into the microscope.
% This arrangement ensured that the higher orders of the beam generated by SLM pixilation were directed to the beam dump.
% The SLM was positioned such that it was in a back focal plane of a long focal length (400 mm) lens, while its focus coincided with the first galvanometric scanner.
% A flip mirror was positioned between first lens in the first beam expander and the 400 mm lens.
% Thus it was possible to bypass the SLM if necessary (e.g. during assembly and testing and when SLM was not operational).
% Another flip mirror was added after the 400 mm lens to couple in the original laser sources through tuneable lenses.
%  The excitation path after the galvanometric scanner remained unchanged.
%
% A time-gated FLIM detection system is available in the Chemical Engineering Department and could be added to the LSFM system.
% The FLIM detection was a single unit containing a low NA beam expander, a time gated image intensifier (LAVision) and a high NA beam expander relaying the intensifier on a camera.
% In the new design it replaced the camera for FLIM imaging.
%
%
% %spt
%
% Move into live cells.
% Neural nets
%
% %homography
%
% A 5 or more point elastic transform could be used.
% issue is that finding 5 exact points is hard. 1024,1024 pixel on camera is center.
%
% \subsection{Expansion Microscopy}
%
% \section{Clearing}
%   %
%   % Optical-clearing optimization of LSFM
%   % Though not specific to LSFM, refractive-index matching by chemical clearing of tissue finds a natural home in this context, which allows for exceptionally large, fixed samples to be imaged with microscopic resolution and in a reasonable period of time.
%   % Nevertheless, the transition to larger sample sizes does provide some unique optical challenges, and although clearing makes even deep tissues accessible, without corresponding changes to the optical components they remain tantalizingly out of reach.
%   % % Dodt et al. reported an ultramicroscope that uses low magnification and NA optics to image cleared mouse brains over centimeter-sized FOVs63 naturally, with some sacrifice to spatial resolution.
%   % The generation of thinner, less-divergent light sheets benefits sub-cellular and macroscopic LSFM imaging alike.
%   % Saghafi et al. were able to shape the illuminating light sheet using several aspheric and cylindrical lenses in series to produce light sheets with a 4-um thickness at the beam waist and with little divergence over several millimeters64.
%   % Others have used binary-pupil phase masks to achieve similar results65,66.
%   % Tomer et al. adopted a different approach to imaging optically cleared tissues in the CLARITY optimized light-sheet microscope (COLM)67, which tiles the acquisition process to cover large FOVs.
%   % The superior collection efficiency afforded by high-NA optics compensates somewhat for the additional exposures by making better use of the available light, whereas the relatively high magnification and NA affords submicron resolution.
%   % To compensate for misalignment caused by residual refractive index inhomogeneities deep inside tissue, an autocalibration routine adjusts the light-sheet position such that the two planes maintain co-alignment throughout the volume.
